\documentclass[10pt,a4paper]{article}
\author{Tim Schmidt}
\title{Algorithmen für planare Graphen}

\usepackage[utf8]{inputenc}
\usepackage{amsmath}
\usepackage{amssymb}
\usepackage{hyperref}
\usepackage{xspace}
\usepackage[german]{babel}
\usepackage[autostyle]{csquotes}
\usepackage[a4paper, total={6in, 8in}]{geometry}
\usepackage{siunitx}
\usepackage{tikz}
\usepackage{subfig}
\usetikzlibrary{shapes, arrows}
\usepackage[activate={true,nocompatibility},final,tracking=true,kerning=true,spacing=true,factor=1100,stretch=10,shrink=10]{microtype}
% activate={true,nocompatibility} - activate protrusion and expansion
% final - enable microtype; use "draft" to disable
% tracking=true, kerning=true, spacing=true - activate these techniques
% factor=1100 - add 10% to the protrusion amount (default is 1000)
% stretch=10, shrink=10 - reduce stretchability/shrinkability (default is 20/20)
\microtypecontext{spacing=nonfrench}

\setlength{\parindent}{0cm}
\setlength{\parskip}{0.2cm}

\graphicspath{{images/}}

% maxwidth parameter for \includegraphics
% see https://tex.stackexchange.com/a/86355
\makeatletter
\def\maxwidth#1{\ifdim\Gin@nat@width>#1 #1\else\Gin@nat@width\fi}
\makeatother

\newcommand{\imageFigure}[4]{%
    \begin{figure}[h]%
        \centering%
        {%
            \setlength{\fboxsep}{1pt}%
            \setlength{\fboxrule}{1pt}%
            %\fbox{
            \includegraphics[width=\maxwidth{#3\textwidth}]{#2}%}%
        }%
        \caption{#1}%
        \label{fig:#4}%
    \end{figure}%
}

\newcommand{\imageFigureMult}[6]{%
    \imageFigureMultS{#1}{#2}{#3}{#4}{#5}{#6}{0.45}%
}

\newcommand{\imageFigureMultS}[7]{%
    \begin{figure}[h]%
        \centering
        \subfloat[#1]{{\includegraphics[width=\maxwidth{#7\textwidth}]{#2} }}%
        \qquad
        \subfloat[#3]{{\includegraphics[width=\maxwidth{#7\textwidth}]{#4} }}%
        \caption{#5}%
        \label{fig:#6}%
    \end{figure}
}

\newcommand{\Kf}{$K_5$}
\newcommand{\Kdd}{$K_{3,3}$}

\begin{document}
	\pagenumbering{Roman}
	{\let\newpage\relax\maketitle}
	\tableofcontents
	\newpage
	\pagenumbering{arabic}
	\setcounter{page}{1}

%%%%%%%%%%%%%%%%%%%%%%%%%%%%%%%%%%%%%%%%%%%%%%%%%%%%%%%%%%%%%%%%%%%%%%%%%%%%%%%

\section{Grundlagen}
Planare Graphen sind Graphen, die sich so in die Ebene einbetten lassen, dass
sich keine Kanten kreuzen.
Eine Einbettung ist dabei eine Zuweisung von Knoten auf Punkte und Kanten auf
Kurven zwischen den inzidenten Knoten im 2D-Raum.

Es gibt Graphen, die nicht planar sind.
Darunter sind der vollständige Graph mit fünf Knoten \Kf und der vollständig
bipartite Graph mit sechs Knoten \Kdd:

\imageFigureMult{\Kf}{k5.png}{\Kdd}{k33.png}{Nicht-planare Graphen}{nonplanar}

\subsection{Allgemeine Aussagen über Knoten- und Kantenmenge}
Über planare Graphen lassen sich folgende allgemeine Aussagen machen:
\begin{itemize}
    \item In einem \textit{zusammenhängenden} planaren Graph erfüllt die Anzahl
        der Knoten $n$, Kanten $m$ und Facetten $f$ den Satz von Euler:
        $$n-m+f=2$$
        Dies lässt sich relativ leicht durch Induktion zeigen:

        In einem einfach zusammenhängenden Pfad gibt es $n$ Knoten, $n-1$
        Kanten und genau eine Facette, nämlich die äußere, also gilt
        $$n-(n-1)+1=2$$

        Wird nun ein Knoten hinzugefügt und mit einer Kante an den Graphen
        angehangen, erhöhen sich $n$ und $m$ jeweils um eins.
        Dabei verändert sich die Summe nicht.
        $$(n+1)-(n-1+1)+1=2$$

        Wird eine Kante zwischen zwei bestehenden Knoten eingefügt, wird damit
        eine Facette durchschnitten.
        Somit erhöht sich $m$ und $f$ um eins und die Summe bleibt weiterhin
        gleich.
        $$n-(n-1+1)+(1+1)=2$$

        \imageFigureMult{$n=4, m=3, f=1$}{line.png}
            {$n=4, m=4, f=2$}{line2.png}
            {Bzgl. Satz von Euler}{eulergraphs}
    \item Ein planarer, \textit{einfacher} Graph mit mindestens 3 Knoten hat
        höchstens $3n-6$ Kanten.
        Wenn er keine Dreiecke hat, sogar nur höchstens $2n-4$ Kanten.

        Dies ist extrem wichtig für die Laufzeiten einiger Algorithmen, da für
        planare Graphen somit $m \in \mathcal{O}(n)$ gilt, während für
        allgemeine Graphen nur $m \in \mathcal{O}(n^2)$ gilt.

        Das lässt sich aus dem Satz von Euler schließen:

        In einem planaren Graph mit maximaler Anzahl von Kanten ist jede
        Facette ein Dreieck (Beweis später).
        Demnach grenzt jede Facette an drei Kanten an.
        Da jede Kante gleichzeitig an zwei Facetten angrenzt gilt:
        $$3f=2m$$
        Eingesetzt in den Satz von Euler:
        \begin{align*}
            n-m+\frac{2}{3}m&=2\\
            n-\frac{1}{3}m&=2\\
            m&=3n-6
        \end{align*}

        Dies gilt für maximale planare Graphen.
        Im Allgemeinen also
        $$m \leq 3n-6$$

        In Graphen ohne Dreiecke ist der Beweis analog mit $4f\leq2m$


    \item Jeder planare Graph enthält einen Knoten mit Grad $\leq 5$

        Sei $n_i$ die Anzahl Knoten mit Grad $i$ und $d_{max}$ der Maximalgrad.
        Dann gilt
        $$n=\sum_{i=0}^{d_{max}}{n_i}$$
        und
        $$2m=\sum_{i=0}^{d_{max}}{i*n_i}$$
        Eingesetzt in die Kantenbegrenzung:
        \begin{align*}
            m &\leq 3n-6\\
            2m &\leq 6n-12\\
            2m+12 &\leq 6n\\
            \sum_{i=0}^{d_{max}}{i*n_i} + 12 &\leq 6*\sum_{i=0}^{d_{max}}{n_i}\\
        \end{align*}
        \begin{align*}
            0*n_0+1*n_1+2*n_2+3*n_3+&4*n_4+5*n_5+6*n_6+7*n_7+\cdots + 12 \leq\\
            6*n_0+6*n_1+6*n_2+6*n_3+&6*n_4+6*n_5+6*n_6+6*n_7\cdots
        \end{align*}
        Subtrahiert man $n_0$ bis $n_6$ nun aus der oberen Zeile raus erhält man
        $$1*n_7+2*n_8+\cdots + 12 ~~~\leq~~~ \underbrace{6*n_0+5*n_1+4*n_2+3*n_3+2*n_4+1*n_5}_{\geq 1\text{, also gibt es mind. einen Knoten in }n_0\text{ bis } n_5}\\$$
        %%
\end{itemize}
% Kanten bei keinen Dreiecken
% Satz von Euler
% Knoten mit Grad <= 5
% Färbbarkeit
\subsection{Triangulierung}
Ein \textit{triangulierter} planarer Graph ist ein Graph, in dem alle Facetten
Dreiecke sind.
Dies ist gleichzeitig ein \textit{maximal planarer} Graph, d.h. es kann keine
weitere Kante hinzugefügt werden, ohne die Planarität zu zerstören.
\begin{itemize}
    \item[\enquote{$\Longrightarrow$}]
        \begin{itemize}
            \item Angenommen Graph ist maximal planar aber nicht trianguliert
            \item Graph nicht trianguliert $\implies$ Es gibt Facette $f$, die
                kein Dreiecke ist
            \item Auf Facette $f$ gibt es Knoten $u, v$, die nicht verbunden
                sind
            \item Kante $\{u, v\}$ kann hinzugefügt werden
            \item Graph ist nicht maximal planar. Widerspruch.
        \end{itemize}
    \item[\enquote{$\Longleftarrow$}]
        \begin{itemize}
            \item Angenommen Graph ist trianguliert aber nicht maximal planar
            \item Es kann eine Kante eingefügt werden, die keine Kreuzung
                verursacht.
            \item Diese Kante muss also durch eine einzige Facette f verlaufen.
            \item Da f ein Dreieck ist, sind alle Knoten in f aber schon
                verbunden. Widerspruch.
        \end{itemize}
\end{itemize}

\subsection{Dualgraphen}
Zu jedem planaren, eingebetteten Graph $G$ gibt es einen \textit{Dualgraph}
$G^*$.
Dieser wird erzeugt, indem jede Facette in $G$ einen Knoten in $G^*$ bekommt
und für jede Kante in $G$ die beiden anliegenden Facetten in $G^*$ verbunden
werden.

\imageFigure{Dualgraph}{dual.png}{.5}{dual}

\textit{Bemerkungen}:
\begin{itemize}
    \item Der Dualgraph muss nicht einfach sein.
        Es treten Multikanten zwischen Dualknoten auf, wenn zwei Facetten in
        $G$ durch mehr als eine Kante verbunden sind.
        Dies ist in triangulierten Graphen offensichtlich nicht möglich.

        Außerdem können Schleifen entstehen, wenn beide Seiten einer Kante in
        $G$ zur gleichen Facette gehören.
        Solche Kanten werden auch \enquote{Brücken} genannt und verbinden zwei
        sonst unzusammenhängende Teilgraphen.
    \item Einige Graphen sind isometrisch zu ihrem Dualgraphen, d.h. man kann
        ihre Dualgraphen so einbetten, dass sie dem Ausgangsgraphen gleichen.
        Solche Graphen nennt man \textit{selbstdual}.
    \item Dualgraphen sind immer abhängig von einer konkreten Einbettung des
        Ausgangsgraphen.
        Ein Graph kann abhängig von der Einbettung verschiedene Dualgraphen
        haben.
\end{itemize}

%%%%%%%%%%%%%%%%%%%%%%%%%%%%%%%%%%%%%%%%%%%%%%%%%%%%%%%%%%%%%%%%%%%%%%%%%%%%%%%

\clearpage
\section{Satz von Kuratowski}
Wie bereits oben vorgestellt gibt es mindestens zwei Graphen, die nicht planar
sind:
Der \Kf~und der \Kdd.

Offensichtlich sind auch all die Graphen nicht-planar, die den \Kf~oder den
\Kdd~als Subgraph enthalten.
Ebenso werden diese beiden Graphen nicht dadurch planar, indem wir einen Knoten
mitten auf einer Kante einfügen.
Ein so entstandener Graph nennt man auch eine \enquote{Unterteilung}.
\imageFigure{Unterteilung von \Kf}{k5-unt.png}{.2}{k5-unt}

Tatsächlich reichen diese zwei Graphen und das Verständnis von Unterteilungen,
um die Menge der planaren Graphen vollständig zu charakterisieren.
% Aussage
\subsection{Kantenkontraktion}
Um eine Unterteilung des \Kf oder \Kdd festzustellen, brauchen wir eine
Technik, um Unterteilungen rückgängig machen zu können.
Diese nennt sich Kantenkontraktion.

Beim Kontrahieren einer Kante $\{u,v\}$ werden die beiden anliegenden Knoten $u$
und $v$ zu einem Knoten $vu$ zusammengelegt und die Kante $\{u,v\}$ gelöscht.
Die restlichen vorher an $u$ und $v$ anliegenden Kanten werden alle auf $vu$
übertragen.
Dadurch entstandene Multikanten werden gelöscht.

\imageFigure{Kontraktion von $\{u,v\}$}{kontr.png}{.7}{kontr}

Entfernt man aus Abbildung \ref{fig:k5-unt} den roten Knoten, indem man eine
der beiden anliegenden Kanten kontrahiert, erhält man wieder genau den \Kf.

Beachte das beim Rückgängigmachen von Unterteilungen immer einer der Knoten,
der an der Kontraktion beteiligt ist, Grad 2 hat.

\subsection{Minor, Topologischer Minor}
Sei $G$ ein Graph.
Wenn es möglich ist den Graphen $H$ per Kantenkontraktion aus einem Subgraphen
von $G$ zu erhalten, dann ist $H$ ein \textit{Minor} von $G$.
Ist dies sogar dann möglich, wenn man nur Kanten kontrahiert, an die ein Knoten
mit Grad 2 anliegt, dann ist $H$ sogar ein \textit{topologischer Minor}.

Da es erlaubt ist, nur einen Subgraphen von $G$ zu betrachten, darf man also
wie folgt vorgehen:
\begin{enumerate}
    \item Entferne beliebig viele Knoten und die anliegenden Kanten aus $G$.
    \item Kontrahiere beliebig viele Kanten in $G$ (beachte obige Beschränkung,
        falls nach topologischen Minoren gesucht wird).
    \item Falls dadurch \Kf~oder \Kdd~rauskommt ist $G$ nicht planar.
\end{enumerate}

\subsection{Satz von Kuratowski / Wagner}
Nach dem \textbf{Satz von Kuratowski} sind \textit{genau} die Graphen planar,
die weder \Kf~noch \Kdd~als topologischen Minor enthalten.

Der \textbf{Satz von Wagner} hingegen besagt, dass \textit{genau} die Graphen
planar sind, die weder \Kf~noch \Kdd~als Minor enthalten.
Diese Minoren müssen nicht zwangsweise topologische Minoren sein.

Folgende Anmerkungen:
\begin{enumerate}
    \item Die Menge der Minoren eines Graphen und die Menge der topologischen
        Minoren eines Graphen sind \textit{nicht} identisch.
    \item Trotzdem sind die Sätze von Kuratowski und Wagner äquivalent.
\end{enumerate}

Beispiel zu Punkt 1:\\
Im Petersengraph in Abbildung \ref{fig:petersen} erhält man \Kf~als Minor,
indem man alle roten Kanten kontrahiert.
Der \Kf~enthält 5 Knoten mit Grad 4.
Durch eine Unterteilung werden die Grade der ursprünglichen Knoten nicht
verändert.
Demnach müsste der Petersengraph auch 5 Knoten mit mindestens Grad 4 besitzen,
um \Kf~als topologischen Minor zu enthalten.
Dies ist nicht der Fall.

Allerdings enthält der Petersengraph den \Kdd~als topologischen Minor.
Die Sätze von Kuratowski und Wagner kommen hier also zum gleichen Ergebnis.

\imageFigure{Petersengraph}{petersen.png}{.7}{petersen}


\subsection{Beweis der Äquivalenz}
Nun ein Beweis zum obigen Punkt 2.

Wir zeigen zuerst, dass wenn \Kdd~als Minor vorkommt, er auch als topologischer
Minor vorkommt, indem wir folgendes Lemma beweisen:

\textbf{Lemma 1.} Wenn der gesuchte Graph $H$ Maximalgrad $\Delta \leq 3$
besitzt und ein Minor von $G$ ist, dann ist er auch ein topologischer Minor von
$G$.

$H$ ein Minor von $G$ und sei $G'$ der Subgraph von $G$, von dem aus wir durch
Kontraktionen auf $H$ kommen.
Angenommen es soll die Kante $\{u,v\}$ kontrahiert werden.
Da in dem Zielgraphen $H$ der Maximalgrad höchstens 3 hat, darf der
resultierende Knoten $uv$ auch maximal Grad 3 besitzen.\footnote{Es gibt
Kontraktionsfolgen, bei denen der Grad von Knoten nachträglich wieder abnimmt.
Keine Ahnung, warum das hier ignoriert werden darf. Es wird im Skript nicht
angesprochen.}

Angenommen $v$ und $u$ haben keine gemeinsamen Nachbarn\footnote{Die
Randfallbehandlung für den anderen Fall im Skript scheint keinen Sinn zu
ergeben, falls der gemeinsame Nachbar Grad 3 hat.}, dann gilt
$$ \delta(vu) = \delta(u) - 1 + \delta(v) - 1 $$

Dies kann aber nur dann höchstens 3 sein, wenn entweder $v$ oder $u$ einen Grad
von höchstens 2 hat.

\imageFigureMultS{$uv$ hat danach Rang 3}{eq-contr.png}{$uv$ hat hier auch Rang 3}{eq-contr2.png}{Kontraktion der Kante $\{u, v\}$}{eq-contr}{.3}

Da einer der Knoten also höchstens Grad 2 hat, darf diese Kontraktion auch
unter topologischer Minorenfindung durchgeführt werden.

$\implies$ Wenn \Kdd~Minor ist, ist er auch topologischer Minor.

Nun zeigen wir noch, dass wenn \Kf~Minor ist, entweder \Kf~oder
\Kdd~topologischer Minor ist.

Sei \Kf~Minor von $G$ und $G'$ der minimale Subgraph, aus dem \Kf~nur durch
Kontraktionen gewonnen wird.
Dann gibt es eine Partition von $G'$ in fünf Knotenmengen $\{V_1, \cdots,
V_5\}$.
Diese sind paarweise miteinander über eine Kante verbunden.
Da $G'$ minimal ist, gibt es keine Kreise innerhalb der Partitionen, weshalb
sie für sich genommen Bäume sind.

Wenn alle $V_i$ Unterteilungen von $K_{1,4}$ sind (wie $V_5$ in Abbildung
\ref{fig:minor-to-top}), dann ist $G'$ eine Unterteilung von \Kf~und hat
demnach \Kf~als topologischen Minor.

Wenn eine Partition keine Unterteilung von $K_{1,4}$ ist, dann muss sie zwei
Knoten mit Grad 3 haben und wir können wir in Abbildung \ref{fig:minor-to-top}b
einen \Kdd-Minor bauen.
Wie oben gezeigt ist dann \Kdd~auch ein topologischer Minor.

$\implies$ Wenn \Kf~ein Minor ist, dann ist \Kf~oder \Kdd~ein topologischer
Minor.

\imageFigureMult{}{minor-to-top1.png}{}{minor-to-top2.png}{}{minor-to-top}

Demnach sind also die Sätze von Kuratowski und Wagner äquivalent.

\subsection{Beweis des Satzes von Wagner}
Es ist einfacher den Satz von Wagner zu beweisen.
Da dieser mit dem Satz von Kuratowski äquivalent ist müssen wir letzteren
danach nicht mehr beweisen.

Dass ein Graph nicht planar sein kann, wenn er einen nicht-planaren Graphen als
Subgraphen oder Minor hat, ist offensichtlich.
Es bleibt also nur die andere Richtung zu beweisen.

Dazu definieren wir \textit{minorminimal nicht planare} Graphen.
Das sind Graphen, die selbst nicht planar sind, aber dessen Minoren alle planar
sind.

Ein solcher Graph hat Minimalgrad 3, da Knoten mit Grad $\leq 2$ kontrahiert
werden können, ohne die Planarität zu ändern, was zu einem nicht-planaren Minor
führen würde.

Offensichtlich ist auch, dass wenn ein Graph $G$ nicht planar ist aber nicht
selbst minorminimal nicht planar ist, einen minorminimal nicht planaren Minor
haben muss.
Das ist leicht zu realisieren wenn man bedenkt, dass man jeden Graphen planar
machen kann, wenn man nur genug Knoten entfernt oder Kanten kontrahiert.
Macht man das so lange, bis jede weitere Entfernung bzw. Kontraktion zu einem
planaren Graphen führen würde, hat man einen minorminimal nicht planaren
Graphen erreicht.

Im folgenden zeigen wir, dass \Kf~und \Kdd~die einzigen minorminimal nicht
planaren Graphen sind.
Daraus folgt dann direkt, dass entweder \Kf~oder \Kdd~enthalten sein muss,
damit ein Graph nicht planar ist.

Dieser Beweis ist ziemlich lang und wir brauchen dafür einiges an Werkzeug, das
wir vorher abarbeiten werden.
Den gesamten Beweis lang haben wir einen Graph $G$, der minorminimal nicht
planar ist und einen Graph $G'$, der entsteht, wenn wir zwei beliebige
\textit{benachbarte} Knoten $x$ und $y$ entfernen.

\textbf{Lemma 3.} $G'$ ist ein einfacher Kreis.

Um Lemma 3 zu beweisen brauchen wir zwei weitere Lemmata:

\textbf{Lemma 5.} $G'$ enthält keinen Theta-Graph (siehe Abbildung
\ref{fig:theta}).\\
\textbf{Lemma 6.} $G'$ enthält höchstens einen Knoten mit Grad $\leq 1$.

\imageFigure{Ein Theta-Graph ist eine beliebige Unterteilung des obigen
Multigraphs}{theta.png}{.3}{theta}

\textbf{Beweis zu Lemma 5}:\\
Folgende Feststellungen sind wichtig:
\begin{itemize}
    \item Nur Bäume und Wälder haben genau eine Facette.
        Bei Graphen mit mehreren Facetten bildet jeder Facettenrand einen
        Kreis.
    \item Sei $G$ planar und $C$ ein einfacher Kreis in $G$.
        Wenn alle Knoten auf $C$ nur durch $C$ verbunden sind, also in $G-E(C)$
        nicht verbunden sind, dann kann man $G$ so einbetten, dass $C$ die
        äußere Facette begrenzt. Siehe Abbildung \ref{fig:c-outer}.
    \item Der Rand $F$ einer Facette $f$, also alle an $f$ anliegenden Knoten
        und Kanten, beinhaltet keinen Theta-Graph. Siehe Abbildung
        \ref{fig:face-no-theta}.
\end{itemize}

\imageFigure{C begrenzt die äußere Facette}{c-outer.png}{.5}{c-outer}
\imageFigure{Facettenränder beinhalten keinen Theta-Graphen}{face-no-theta.png}{.25}{face-no-theta}

Da $G$ minorminimal nicht planar ist, muss $G'$ planar sein.
Wir können also eine planare Einbettung $\Gamma_{G'}$ von $G'$ erstellen.
Wir können sie sogar so konstruieren, dass alle zu $x$ oder $y$ benachbarten
Knoten auf dem Rand der gleichen Facette $f$ liegen.

Wenn wir jetzt den Facettenrand von $f$ betrachten, dann enthält dieser
nach obiger Beaobachtung keinen Theta-Graphen.
Angenommen $G'$ enthält einen Theta-Graph, dann ist $G'$ kein Baum oder Wald
und der Facettenrand von $f$ beinhaltet einen Kreis $C$.

Wir können die Einbettung $\Gamma_{G'}$ so konstruieren, dass auch die
entfernten Knoten $x$ und $y$ im Inneren des Kreises $C$ liegen.\footnote{Das
Skript benutzt hier die Notation $G/xy$, die nicht weiter erklärt wird.
Möglicherweise ist $xy$, also die Kontraktion der Kante $\{x, y\}$ gemeint.}

Da $G'$ einen Theta-Graphen enthält, muss sich mindestens noch eine weitere
Kante $e$ außerhalb des Facettenrandes, also in $E(G') \backslash E(F)$,
befinden.
Damit liegt diese Kante auch außerhalb des Kreises $C$, also in $ext(C)$.

Da $G$ minorminimal nicht planar und $ext(C)$ nicht leer ist, muss $G-ext(C)$
planar sein.
Da auch hier innerhalb von $C$ kein Theta-Graph besteht und demnach keine zwei
Knoten auf $C$ durch eine weitere Kante innerhalb von $C$ verbunden sind, kann
man wie in der zweiten obigen Beobachtung $G-ext(C)$ so einbetten, dass $C$
die äußere Facette begrenzt.\footnote{Aber warum sollte $C$ auf $G-ext(C)$
keinen Theta-Graphen enthalten? Diese Folgerung basierte auf $G'$. Hier ist $x$
und $y$ ja noch enthalten, also sollte es doch auch möglich sein, dass $C$
nicht Teil eines Facettenrandes ist.}

Wenn wir jetzt die Einbettungen $\Gamma_{G'}$ und $\Gamma_{G-ext(C)}$ jeweils
bei $C$ teilen und dann kombinieren, haben wir eine planare Einbettung von $G$.
Widerspruch.

$\implies$ $G'$ enthält keinen Theta-Graphen. (Lemma 5)

\textbf{Beweis zu Lemma 6}:\\
Zu beweisen ist, dass $G'$ höchstens einen Knoten mit Grad $\leq 1$ besitzt.

Knoten von Grad 0 sind nicht möglich:
$G$ hat Minimalgrad 3.
Durch entfernen von $x$ und $y$ kann der Grad eines Knotens höchstens auf 1
sinken, aber nicht auf 0.

Angenommen es gäbe zwei Knoten $v$ und $u$ in $G'$ mit Grad 1.
Dann müssten in $G$ beide adjazent zu $x$ und $y$ sein.
Da die restlichen Knoten in $G$ auch Minimalgrad 3 haben, sind $v$ und $u$ noch
über einen weiteren Weg $p$ verbunden.
Dadurch enthält $G$ einen Theta-Graph in $\{x,y,u,v\} \cup p$.

In diesem Fall darf es keine einzige Kante $\{i, j\}$ geben, die nicht an $\{x,
y, u, v\}$ inzident ist.
Ansonsten würde $G - i - j$ einen Theta-Graphen enthalten, was nach Lemma 5
nicht möglich ist.

\imageFigure{Es darf keine unabhängige Kante geben.}{xuvy1.png}{.3}{xuvy1}

Damit gibt es nur noch weniger Möglichkeiten für das Aussehen von $G$:

\imageFigure{
    $u$ und $v$ sind verbunden.
    Es gibt keine weiteren Knoten.
}{xuvy2.png}{.3}{xuvy2}

\imageFigure{
    Es gibt den weiteren Knoten $z$.
    $u$ und $v$ können nicht verbunden sein, da sie in $G$ höchstens Grad 3
    haben dürfen.
}{xuvy3.png}{.9}{xuvy3}

\imageFigure{
    Es gibt zwei weiteren Knoten $z$ und $z'$.
    Der rechte Graph ist eine planere Einbettung des linken Graphs.
    $u$ und $v$ können nicht verbunden sein, da sie in $G$ höchstens Grad 3
    haben dürfen.
    $z$ und $z'$ dürfen nicht verbunden sein, da sie eine unabhängige Kante
    bilden würden.
    Einen zusätzlichen Knoten $z''$ kann man nicht einfügen, da dieser
    mindestens Grad 3 haben müsste.
    Zu $v$ und $u$ dürfte er wegen des Maximalgrad von $v$ und $u$ nicht
    verbunden sein und zu $z$ und $z'$ nicht, weil es eine unabhängige Kante
    wäre.
}{xuvy4.png}{.7}{xuvy4}

Siehe Abbildungen \ref{fig:xuvy2}, \ref{fig:xuvy3} und \ref{fig:xuvy4}.
Demnach wäre $G$ auf jeden Fall planar, was einen Widerspruch darstellen würde.

$\implies$ $G'$ hat maximal einen Knoten mit Grad $\leq$ 1. (Lemma 6)

\clearpage
\textbf{Beweis zu Lemma 3}:\\
Wir beweisen nun, dass $G'$ ein einfacher Kreis ist.
Dazu nehmen wir an, dass $G'$ \textit{kein} einfacher Kreis ist und
schlussfolgern daraus den Widerspruch, dass $G$ planar ist.

Dazu definieren wir erst einmal die \textit{Blockzerlegung}.
Zwei Kanten $e_1$ und $e_2$ sind äquivalent im Sinne der Blockzerlegung, wenn
sie entweder gleich sind oder es einen einfachen Kreis gibt, der beide enthält.

Die so induzierte Äquivalenzklassenzerlegung sieht man bspw. in Abbildung
\ref{fig:block1}.
Jede Kante ist dabei in genau einem Block und es gibt möglicherweise Knoten,
die in mehreren Blöcken enthalten sind.
Diese nennt man Seperatorknoten, hier rot dargestellt.

\imageFigure{Blockzerlegung}{block1.png}{.7}{block1}

Nach Lemma 5 enthält $G'$ keinen Theta-Graphen, weshalb alle Blöcke entweder
einzelne Kanten oder einfache Kreise sind.
Auch ist es nicht möglich, dass alle Blöcke in einem Kreis angeordnet sind,
weil sie sonst zu einem Block zusammenfallen würden.
Es muss also \textit{Endblöcke} geben.

Nach Lemma 6 gibt es maximal einen Knoten mit Grad $\leq 1$.
Da $G'$ nach Vorraussetzung kein einfacher Kreis ist und es nach Lemma 6 auch
keine einzelne Kante ist, muss es mindestens zwei Blöcke und demnach auch zwei
Endblöcke geben.
Maximal einer dieser Endblöcke kann eine einfache Kante sein.
Der anderen muss ein Kreis sein.

\imageFigure{Blockzerlegung von $G'$}{block2.png}{.7}{block2}

Diesen Kreis nennen wir jetzt $C$ und den Separatorknoten darauf $v$.
Da alle anderen Knoten auf $C - v$ in $G'$ Grad 2 haben, aber $G$ Minimalgrad 3
hat, müssen diese Knoten zu $x$ oder $y$ adjazent sein.
Da ein Kreis mindestens Länge 3 hat, gibt es mindestens zwei andere Knoten auf
$C - v$.
Diese bilden wie in Abbildung \ref{fig:prisma1} zu sehen ist einen Theta-Graph
mit $\{x, y, v\}$.

\imageFigure{Alle anderen Knoten auf $C-v$ müssen zu $x$ oder $y$ adjazent
sein.
Alternative Kanten sind in Grau dargestellt.
}{prisma1.png}{.5}{prisma1}

Nach Lemma 5 gibt es keine unabhängige Kante, also keine Kante außerhalb von
$\{x, y\} \cup C$, da ihr Entfernen aus $G$ ein $G'$ mit Theta-Graph
produzieren würde.
Isolierte Knoten sind nach Lemma 6 auch unmöglich, weshalb es höchstens einen
weiteren Knoten $u$ geben kann.
Sieh Abbildung \ref{fig:prisma2}.

\imageFigure{Es gibt höchstens $u$ als zusätzlichen
Knoten.}{prisma2.png}{.4}{prisma2}

Da aber auch $G-u-v$ keinen Theta-Graph enthalten darf, muss $C$ Länge 3 haben.
Siehe Abbildung \ref{fig:prisma3}a.

Der resultierende Graph mit $|C| = 3$ ist aber der sogenannte Prisma-Graph.
Dieser ist planar. Widerspruch.

\imageFigureMultS{$C$ kann nicht Länge $> 3$ haben.}{prisma3.png}
{\textit{Prisma}-Graph}{prisma3.png}{}{prisma3}{.35}

$\implies$ $G'$ ist ein einfacher Kreis.

Nun können wir uns in Richtung des \textbf{Satzes von Wagner} bewegen.
Aus Lemma 3 folgern wir, dass $G$ aus einem Kreis $C$, zwei Knoten $x$ und $y$,
der Kante $\{x,y\}$ und Kanten zwischen $C$ und $x$ und $y$ besteht.

Vorher beweisen noch schnell ein kleines Lemma:

\textbf{Lemma 4}. Für zwei benachbarte Knoten $u$ und $v$ auf $C$ gilt:
Wenn $u$ zu exakt einem von $\{x, y\}$ verbunden, ist, dann ist $v$ nicht zu
diesem Knoten aus $\{x, y\}$ verbunden.
Formaler:
$$[~\{u,x\} \in E \land \{u,y\} \notin E~] \implies \{v,x\} \notin E$$

Sei o.B.d.A. und nur zur Veranschaulichung angenommen, dass $G$ aussieht wie in
Abbildung \ref{fig:lem4}.
Würde es die Kante $\{v, x\}$ geben, dann könnte sie entfernt werden, um einen
planaren Graphen aus $G$ zu erzeugen, da $G$ minorminimal nicht planar ist.

Wir können diese Kante aber danach wieder planar einfügen:
Knoten $u$ ist nur durch zwei Kanten an $C$ angeschlossen und durch eine Kante
an $x$.
Das bedeutet, dass unabhängig von der Einbettung die Kanten $\{u, v\}$ und
$\{u, x\}$ direkt aufeinander folgen.
Die Kante $\{v, x\}$ produziert also keine Kreuzung.
Damit ist $G$ planar einbettbar. Widerspruch.

$\implies$ Lemma 4 bewiesen.

\imageFigure{Die Kante $\{v, x\}$ darf es hier nicht
geben.}{lem4.png}{.4}{lem4}

Da $G$ Minimalgrad 3 hat, muss jeder Knoten auf $C$ mit mindestens einem von
$\{x, y\}$ verbunden sein.
Nach Lemma 4 sind dann entweder alle Knoten auf $C$ mit $x$ \textit{und} $y$
verbunden, oder sie sind abwechselnd mit nur einem davon verbunden.
Ersteres klingt schon sehr nach \Kf~und letzteres klingt nach der Zweifärbung,
die wir in \Kdd~haben.

In einem letzten Schritt begrenzen wir noch die Größe von $C$.
Angenommen $C$ hat Größe $\geq 5$.
Siehe Abbildung \ref{fig:csize}a.
Entfernt man zwei beliebige auf $C$ benachbarte Kanten $i$ und $j$, dann
verbleiben die Kanten $x, y, v_1, v_2, v_3$, wobei $v_2$ zwischen $v_1$ und
$v_3$ auf $C$ liegt.

\imageFigureMultS{$|C| \geq 5$}{csize1.png}
{$|C| = 4$ und alle Knoten sind mit $x$ und $y$ verbunden.}{csize2.png}
{Hinweis: Es sind nicht alle Kanten eingezeichnet.}{csize}{.2}

Da $v_2$ zu $v_1$ und $v_3$, aber auch zu einem von $\{x, y\}$ adjazent sein
muss\footnote{Warum?}, hat $v_2$ Grad 3, womit $G - i - j$ kein einfacher Kreis
ist. Widerspruch.

$\implies$ $C$ hat Größe 3 oder 4.

Falls $C$ Größe 4 hat und alle Knoten mit $x$ \textit{und} $y$ verbunden sind
folgt der gleiche Widerspruch.
Siehe Abbildung \ref{fig:csize}b.

$\implies$ Entweder $C$ hat Größe 3 oder $C$ hat Größe 4 und die Knoten sind
abwechselnd mit $x$ und $y$ verbunden.

Die abwechselnde Verbundenheit mit $x$ und $y$ ist auf einem Kreis der Größe 3
offensichtlich nicht möglich, weshalb im Falle $|C|=3$ alle Knoten mit $x$ und
$y$ verbunden sind.

Diese zwei Möglichkeiten ergeben genau den \Kf~und den \Kdd.
Siehe Abbildung \ref{fig:proof-k5}.

$\implies$ \Kf~und \Kdd~sind die einzigen minorminimal nicht planaren Graphen.

$\implies$ Alle nicht planaren Graphen beinhalten \Kf~oder \Kdd~als Minor.

\imageFigureMultS{\Kf}{proof-k5.png}{\Kdd}{proof-k33.png}{}{proof-k5}{.3}

%%%%%%%%%%%%%%%%%%%%%%%%%%%%%%%%%%%%%%%%%%%%%%%%%%%%%%%%%%%%%%%%%%%%%%%%%%%%%%%

\clearpage
\section{Färbung}
Jeder planare Graph ist \textit{4-färbbar}.
Der Beweis dafür ist allerdings sehr aufwändig und nicht im Rahmen der
Vorlesung.
Stattdessen wird die \textit{5-Färbbarkeit} bewiesen.

\subsection{5-Färbbarkeit}
Der Beweis läuft per Induktion über die Knotenanzahl $n$ des Graphen $G$.

\begin{itemize}
    \item[\textbf{IA}:] Für $n \leq 4$ ist gilt die Behauptung trivialerweise.

    \item[\textbf{IV}:] Alle planaren Graphen mit $n-1$ Knoten sind 5-färbbar.

    \item[\textbf{IS}:] $n-1 \rightarrow n$\\ Da $G$ planar ist gibt es einen
        Knoten $v$ mit Grad $\leq 5$.

        \begin{itemize}
            \item[\textbf{Fall 1}:] $v$ hat Grad $\leq 4$.
                Dann hat $v$ die Nachbarn $w_1, \dots, w_4$.
                Da $G-v$ nach IV 5-färbbar ist, existiert eine 5-Färbung von
                $G-v$.
                In dieser 5-Färbung werden von $w_1, \dots,w_4$ nur 4 von 5
                Farben belegt.
                Die übrige Farbe wird v zugeordnet.

                $\implies$ $G$ ist 5-färbbar.
            \item[\textbf{Fall 2}:] $v$ hat Grad 5.
                Dann hat $v$ die Nachbarn $w_1, \dots, w_5$.
                Die Reihenfolge der Nachbarn ist hier wichtig.
                Falls die Nachbarn nicht alle Farben ausnutzen, weise die
                übrige Farbe $v$ zu.

                Ansonsten haben alle Nachbarn verschiedene Farben.
                Betrachte nun $w_1$ und $w_4$.
                Wenn sie in $G-v$ nicht verbunden sind, dann liegen sie in den
                getrennten Zusammenhangskomponenten $H_1$ und $H_4$.
                Wenn nun in $H_4$ die Farbe 1 und 4 vertauscht wird, dann ist
                die Färbung in $H_4$ immer noch korrekt, aber $w_4$ hat nun
                Farbe 1.
                Danach ist für $v$ die Farbe 4 übrig.

                Sollten $w_1$ und $w_4$ in $G-v$ nicht in getrennten
                Zusammenhangskomponenten liegen, dann betrachte stattdessen
                $w_3$ und $w_5$.
                Wie in Abbildung \ref{fig:color} zu sehen ist, kann in dem
                planaren Graphen $G-v$ nicht $w_1$ mit $w_4$ \textit{und} $w_3$
                mit $w_5$ verbunden sein.

                $\implies$ $G$ ist 5-färbbar.

                \imageFigure{Beachte dass wir absichtlich Nachbarpaare wählen,
                die nicht direkt nebeneinander liegen.}{color.png}{.25}{color}
        \end{itemize}
\end{itemize}


\subsection{5-Listenfärbbarkeit}
TODO % TODO
\subsection{4-Listenfärbbarkeit}
TODO % TODO

% Verbindung Maximalgrad <-> Chromatische Zahl

%%%%%%%%%%%%%%%%%%%%%%%%%%%%%%%%%%%%%%%%%%%%%%%%%%%%%%%%%%%%%%%%%%%%%%%%%%%%%%%

\clearpage
\section{Kleinkram}
\subsection{Außenplanare Graphen}
\begin{itemize}
    \item[] $G$ ist \textit{außenplanar}.
    \item[$\iff$] $G$ lässt sich so planar einbetten, dass alle Knoten auf der
        äußeren Facette liegen.
    \item[$\iff$] Man kann einen Knoten zu $G$ hinzufügen und mit allen
        vorhandenen Knoten verbinden, ohne die Planarität von $G$ zu zerstören.
    \item[$\iff$] $G$ enthält keine Unterteilung von $K_4$ oder $K_{2,3}$.
\end{itemize}
Ein außenplanarer Graph hat außerdem höchstens $2n-3$ Kanten.

Betrachte Abbildung \ref{fig:outplane}.
Wenn $G$ eine Unterteilung von $K_4$ oder $K_{2,3}$ enthalten würde, würde
durch das Hinzufügen von $v$ eine Unterteilung von \Kf~bzw. \Kdd~entstehen.

Andersherum kann man auch argumentieren, dass um durch Hinzufügen von $v$ ein
\Kf~oder \Kdd~zu produzieren, vorher offensichtlich bereits eine Unterteilung
des $K_4$ bzw. $K_{2,3}$ vorhanden gewesen sein muss.

\imageFigure{}{outplane.png}{.4}{outplane}

Zur Kantenbegrenzung:\\
$G$ hat $n$ Knoten und $m$ Kanten.
$G+v$ hat $n'=n+1$ Knoten und $m'=m+n$ Kanten.
\begin{align*}
    &G+v \text{ ist planar.}\\
    \implies &m' \leq 3n'-6\\
    \implies &m+n \leq 3(n+1)-6\\
    \implies &m \leq 2n-3
\end{align*}


\subsection{Petersengraph}
Der Petersengraph besteht aus folgenden Knoten und Kanten:
Die Knoten sind alle zweielementigen Teilmengen von $\{1,2,3,4,5\}$.
Die Kanten bestehen zwischen allen Knoten $u$ und $v$, die keine gemeinsamen
Zahlen besitzen.
\imageFigure{Petersengraph}{petersen2.png}{.4}{petersen2}
\subsection{Adjazenztest in $\mathcal{O}(1)$}
\begin{itemize}
    \item Start mit ungerichtetem Graphen.
    \item Es gibt einen Knoten $v$ mit (Ausgangs-)Grad $\leq 5$.
    \item Orientiere alle nicht-orientierten Kanten von v weg.
    \item Es gibt einen neuen Knoten mit Ausgangsgrad $\leq 5$.
    \item Mach mit dem das gleiche.
    \item Am Ende haben alle Knoten Ausgangsgrad $\leq 5$.
    \item $Adjazent(v,u) = v\in\mathcal{N}^+(u) \lor u\in\mathcal{N}^+(v)$
    \item Da $\mathcal{N}^+(x)$ begrenzt, ist $Adjazent \in \mathcal{O}(1)$.
\end{itemize}
\subsection{Core-Zerlegung}
\subsection{Entartetheit}
\subsection{Dreiecke zählen}
\begin{itemize}
    \item Nutze Adjazenztest in $\mathcal{O}(1)$
    \item Für alle $v$, prüfe ob zwei Nachbarn adjazent sind.
    \item In $\mathcal{O}(n)$.
\end{itemize}
\imageFigure{Algorithmus zum Dreiecke zählen}{dreiecke.png}{.5}{dreiecke}

\subsection{MST}
Siehe Abbildung \ref{fig:mst}. Erwartet $\mathcal{O}(n)$.
\imageFigure{Algorithmus zur Konstruktion eines MST}{mst.png}{.8}{mst}

%%%%%%%%%%%%%%%%%%%%%%%%%%%%%%%%%%%%%%%%%%%%%%%%%%%%%%%%%%%%%%%%%%%%%%%%%%%%%%%

\clearpage
\section{Planar-Separator-Theorem}
In jedem planaren Graphen $G$ mit $n \geq 5$ gibt es eine Knotenmenge $S$,
genannt \textit{Separator}, die bei Herausnahme $G$ in zwei
Zusammenhangskomponenten $V_1, V_2$ zerlegt, sodass

\begin{itemize}
    \item $|V_1| \leq \frac{2}{3}n$
    \item $|V_2| \leq \frac{2}{3}n$
    \item $|S| \leq 4\sqrt{n}$
\end{itemize}

Dafür brauchen wir zuerst ein

\textbf{Lemma 4.2} Sei $T$ ein aufspannender Baum in $G$ mit Wurzel $w$ und
Höhe $h$, dann kann $G$ so in $V_1, V_2, S$ partitioniert werden, dass gilt:
\begin{itemize}
    \item $|V_1| \leq \frac{2}{3}n$
    \item $|V_2| \leq \frac{2}{3}n$
    \item $|S| \leq 2h+1$
\end{itemize}

\textbf{Beweis zu Lemma 4.2}:\\
Sei $G'$ die Triangulierung von $G$.
Dann hat $G$ nach dem Satz von Euler $3n-6$ Kanten und $2n-4$ Facetten.
Offensichtlich ist $T$ auch ein aufspannender Baum von $G'$.

Jede Kante in $G'$ ist entweder Teil von $E(T)$, also eine \textit{Baumkante},
oder eine \textit{Nichtbaumkante}.
Jede Nichtbaumkante $\{x,y\}$ verläuft zwischen zwei Knoten im Baum $T$ und
induziert damit einen Kreis, der über den gemeinsamen Vorfahren von $x$ und $y$
verläuft.
Dieser Kreis hat maximal Länge $2h+1$ und kann somit als Separator verwendet
werden und teilt $G$ in $\text{Inneres}(K_{x,y})$ und
$\text{Äußeres}(K_{x,y})$.
Wir müssen also nur noch die Größe dieser beiden Partitionen anpassen.

Sei jetzt o.B.d.A das Innere größer.
Dann müssen wir nur dafür sorgen, dass das Innere $\leq \frac{2}{3}n$ ist.
Wir behandeln hier naheliegenderweise nur den Fall, dass das Innere $>
\frac{2}{3}n$ ist.

Im triangulierten Graph $G'$ begrenzt die Kante $\{x, y\}$ zwei Dreiecke, von
denen eines im Inneren liegt.
Die dritte am inneren Dreiecke beteiligte Knoten sei nun $t$ genannt.
Die Kanten $\{x,t\}$ und $\{y,t\}$ können entweder Baumkanten oder
Nichtbaumkanten sein.

Falls eine der beiden eine Baumkante ist, o.B.d.A. $\{y, t\}$, dann ersetze
$\{x, y\}$ durch $\{x, t\}$.
Dadurch gilt:

Falls $t$ auf dem Kreis liegt:
Das Äußere bleibt gleich groß, das Innere eins kleiner.
Siehe Abbildung \ref{fig:pst-tree}.

Falls $t$ nicht auf dem Kreis liegt:
Das Äußere wird um eins größer und das Innere bleibt gleich.

In diesem Fall wird also das Verhältnis zwischen Innerem und Äußerem höchstens
in die richtige Richtung geschoben und nicht überkorrigiert.

\imageFigure{}{pst-tree.png}{.8}{pst-tree}

Sollten aber beide Kanten Nichtbaumkanten sein, dann können wir statt $K_{x,y}$
den Kreis $K_{x,t}$ oder $K_{y,t}$ verwenden.
Siehe Abbildung \ref{fig:pst-tree2}.
Dazu müssen wir die Größe von $\text{Inneres}(K_{x,t})$ und
$\text{Inneres}(K_{y,t})$ vergleichen.
Sei hier o.B.d.A $\text{Inneres}(K_{x,t}) \geq \text{Inneres}(K_{y,t})$.

\imageFigure{}{pst-tree2.png}{.4}{pst-tree2}

Wenn wir nun $\{x,y\}$ durch $\{x,t\}$ ersetzen, gilt
\begin{align*}
    \text{Inneres}(K_{x,t}) &\leq \text{Inneres}(K_{x,y}) - 1\\
    \text{Äußeres}(K_{x,t}) &\leq n - \frac{1}{2}\text{Äußeres}(K_{x,y}) \leq
    \frac{2}{3}n
\end{align*}

Inneres wird also auf jeden Fall kleiner und Äußeres wird nicht zu groß.

Das wiederholen wir solange, bis Inneres klein genug ist.

$\implies$ Es kann ein gültiger Separator der Größe $\leq 2h-1$ gefunden
werden.

Für den Beweis des Satzes verwenden wir noch folgendes

\textbf{Lemma 4.3}. Sei $T$ ein BFS-Baum zum Graphen $G$.
Jede Nichtbaumkante $\{u, v\}$ in $G$ verbindet zwei Knoten in $T$, die
entweder auf dem selben oder in anliegenden Leveln sind:
$$ |level(u) - level(v)| \leq 1 $$

Beweis im Skript.
Hier ausgelassen.

Sei $T$ der BFS-Baum von $G$ und $G'$ eine Triangulierung von $G$.
Die Level von $T$ seien mit $0,1,\dots,h$ nummeriert und $S_i$ bezeichnet die
Knotenmenge in Level $i$.

Dann gehe die Level soweit runter, bis du mehr als die Hälfte der Knoten hinter
dir hast.
Das aktuelle Level nenne dann $\mu$.

Falls $|S_\mu| \leq 4\sqrt{n}$, dann sind wir fertig und $S_\mu$ ist unser
Separator.

Ansonsten wähle mit $m$ das nächste obere Level und mit $M$ das nächste untere
Level mit Größe $\leq \sqrt{n}$.
Siehe Abbildung \ref{fig:pst-levels}.

\imageFigure{}{pst-levels.png}{.8}{pst-levels}

Dann setze $A_1$ gleich alle Level über $m$, $A_2$ gleich alle Level unter $M$
und $A_2$ gleich die dazwischen, exkl. $m$ und $M$.

Falls $|A_2| \leq \frac{2}{3}n$, dann wähle $S = S_m \cup S_M$, $V_1$ als
Maximum von $A_1, A_2, A_3$ und $V_2$ als die restlichen $A_i$.
Da $V_1$ als Maximum der $A_i$ gewählt ist, ist $V_2$ höchstens doppelt so groß
wie $V_1$ und demnach $\leq \frac{2}{3}n$.

Sei stattdessen $|A_2| > \frac{2}{3}n$.
Dann ist innerhalb von $A_2$ die Höhe maximal $\sqrt{n}$ und es gibt nach Lemma
4.2 eine Zerteilung von $A_2$ in $S', V'_1, V'_2$ mit $|S'| \leq 2\sqrt{n}+1$
und $|V'_i| \leq \frac{2}{3}|A_2|$.

Setze dann $S = S' \cup S_m \cup S_M$, dadurch $|S| \leq 4\sqrt{n}+1$.
Wähle danach $V_1$ als das Maximum von $V'_1, V'_2$ und wähle $V_2$ als $A_1
\cup A_3 \cup min(V'_1, V'_2)$.
Siehe Abbildung \ref{fig:pst-final}.


\imageFigure{}{pst-final.png}{.95}{pst-final}


%%%%%%%%%%%%%%%%%%%%%%%%%%%%%%%%%%%%%%%%%%%%%%%%%%%%%%%%%%%%%%%%%%%%%%%%%%%%%%%

\clearpage
\section{Planaritätstest mit PQ-Baum}
TODO % TODO

%%%%%%%%%%%%%%%%%%%%%%%%%%%%%%%%%%%%%%%%%%%%%%%%%%%%%%%%%%%%%%%%%%%%%%%%%%%%%%%

\clearpage
\section{Triangulierung}
TODO % TODO

%%%%%%%%%%%%%%%%%%%%%%%%%%%%%%%%%%%%%%%%%%%%%%%%%%%%%%%%%%%%%%%%%%%%%%%%%%%%%%%

\clearpage
\section{Matchings}
Ein \textit{Matching} eines Graphen $G$ ist eine Teilmenge $M$ der Kanten, in
der keine zwei Kanten zum gleichen Knoten inzident sind.
Jeder Knoten, der an einer Kante aus $M$ liegt, heißt \textit{gematcht}.
Andernfalls heißt er \textit{ungematcht}.

Bei dem \textbf{Matching-Problem} will man ein Matching maximalen Gewichts
finden.
Ist das Kantengewicht überall ein, ist gleichzeitig ein Matching maximaler
Kardinalität gesucht.

Ein bezüglich $M$ \textit{alternierender} Weg ist ein Weg oder Kreis, dessen
Kanten abwechselnd in und außerhalb von $M$ liegen.
In einigen Fällen kann man alle Kanten eines alternierendes Weges flippen und
somit ein neues Matching erhalten.
Falls dieses neue Matching schwerer ist, dann heißt der alternierende Weg auch
\textit{erhöhender Weg}.

Um flippen zu können muss der Weg entweder ein Kreis gerader Länge sein oder
die erste und letzte Kante jeweils in $M$ oder inzident zu einem ungematchten
Knoten sein.

\textbf{Lemma 5.1}. Ein Matching $M$ von $G$ hat genau dann maximales Gewicht,
wenn es bzgl. $M$ keinen erhöhenden Weg gibt.

Die Rückrichtung ist offensichtlich.
Wir beweisen die Hinrichtung:
Sei $M$ ein nicht-maximales Matching, für das es keinen erhöhenden Weg gibt.
Dann gibt es ein schwereres, maximales Matching $M^*$.
Betrachte die Differenz der beiden Matchings $M\Delta M^*$.
Dies ist die Menge aller Kanten, die in genau einem von $\{M, M^*\}$ vorkommen.

Da an Kreuzungen maximal zwei Kanten in $M\Delta M^*$ liegen können, hat jeder
Knoten in $M\Delta M^*$ Grad 1 oder 2.
Die Differenz besteht also aus einfachen Wegen und Kreisen.

TODO %TODO

\subsection{Algorithmus}
\begin{itemize}
    \item Falls $n \leq 3$, nimm einfach die größte Kante und fertig.
    \item Sonst zerlege $G$ in $V_1, V_2, S$.
        Führe Algorithmus auf $V_1$ und $V_2$ aus um Matchings $M_1, M_2$ zu
        bekommen.
        Setze dann $M = M_1 \cup M_2$.
        Sei $V' = V_1 \cup V_2$ die Knotenmenge ohne Separator.
    \item Solange der Separator nicht leer ist, verschiebe einen Knoten von $S$
        nach $V'$ und finde ein neues maximales Matching nach Lemma 5.2
\end{itemize}

Laufzeit in $\mathcal{O}(n^\frac{3}{2})$ falls der Lemma-5.2-Schritt in
$\mathcal{O}(n)$.\\
Laufzeit in $\mathcal{O}(n^\frac{3}{2} * \log n)$ falls der Lemma-5.2-Schritt
in $\mathcal{O}(n * \log n)$.

%%%%%%%%%%%%%%%%%%%%%%%%%%%%%%%%%%%%%%%%%%%%%%%%%%%%%%%%%%%%%%%%%%%%%%%%%%%%%%%

\clearpage
\section{Mixed-Max-Cut und Via-Minimierung}
Das \textbf{Min-Cut-Problem} liegt für allgemeine Graphen in $\mathcal{P}$.
Das \textbf{Max-Cut-Problem} ist für allgemeine Graphen $\mathcal{NP}$-schwer.

Das allgemeinere \textbf{Mixed-Max-Cut-Problem} ist für allgemeine Graphen
natürlich auch $\mathcal{NP}$-schwer.
Auf planaren Graphen ist jedoch Mixed-Max-Cut-Problem in
$\mathcal{O}(n^\frac{3}{2} * \log n)$ lösbar.

Auf planaren Graphen ist das Mixed-Max-Cut-Problem äquivalent zum
Mixed-Max-Kreis-Problem auf dem Dualgraphen:

\textbf{Mixed-Max-Kreis-Problem}\\
Finde auf einem Graphen $G$ eine maximal schwere \textit{gerade} Kantenmenge.

\textit{Gerade} bedeutet hierbei, dass alle Knoten geraden Grad haben.

Ein Matching, dass genau $\frac{n}{2}$ Knoten matcht, heißt \textit{perfektes
Matching}.

\subsection{Algorithmus von Shih, Wo und Kuo}
\begin{enumerate}
    \item Trianguliere $G$ zu $G_T$ und gebe den neuen Kanten Gewicht 0.
    \item Berechne den Dualgraphen $G^*$ zu $G_T$.
        Dabei hat jede Kante das Gewicht ihrer Dualkante.
        Da $G_T$ trianguliert ist, hat in $G^*$ jeder Knoten Grad 3.
    \item Kontruiere $G'$ aus $G$ mit nachfolgenden Algorithmus.
        Ein perfektes minimales Matching in $G'$ induziert eine gerade Menge
        maximalen Gewichts in $G^*$.
    \item Konstruiere eben dieses perfekte minimale Matching in $G'$ mit
        weiterem nachfolgenden Algorithmus.
    \item Falls bei Rückführung von $M$ auf $G^*$ eine nichtleere Teilmenge von
        $E^*$ entsteht, dann gebe den dualen Schnitt dieser Teilmenge in $G$
        aus.
        Ansonsten verwende weiteren nachfolgenden Algorithmus zum konstruieren
        einer alternativen maximalen geraden Menge in $G^*$.
\end{enumerate}

\textbf{Algorithmus zu Schritt 3}:\\
Alle Knoten in $G^*$ haben Grad 3.
Ersetze alle Knoten nach dem Schema in Abbildung \ref{fig:shih1}.
Alle neuen Kanten erhalten dabei Gewicht 0.

\imageFigure{}{shih1.png}{.6}{shih1}

Alle Knoten in $G^*$ haben Grad 3, also gibt es eine gerade Anzahl Knoten.
Dies gilt auch nach allen Ersetzungen für $G'$.
Es existiert also ein perfektes Matching auf beiden Graphen.

\textbf{Lemma 6.3}. Falls M ein perfektes Matching auf $G'$ ist, dann
induzieren die restlichen Kanten $E'\backslash M$ eine gerade Menge $M^*$ in
$G^*$.
Andersrum induziert eine gerade Menge $E^*_o$ in $G^*$ ein perfektes Matching
in $G'$.

\textbf{Beweis zu Lemma 6.3}\\
\enquote{$\implies$}:\\
Wie in Abbildung \ref{fig:shih2} zu sehen ist, gibt es pro Knoten zwei mögliche
Matchings.
Beide induzieren in $G^*$ eine Kantenmenge, mit der $v$ geraden Grad hat.
Im linken Fall hat $v$ Grad 0, im rechten Grad 2.

\imageFigure{}{shih2.png}{.7}{shih2}

\enquote{$\Longleftarrow$}:\\
In $E^*_o$ ist der Grad von $v$ entweder 0 oder 2.
Bei Grad 0 ist die linke Konfiguration aus Abbildung \ref{fig:shih2} möglich,
bei Grad 2 die rechte Konfiguration.
Die rechte Konfiguration ist auch mit einer anderen Belegung von $\{e_1, e_2,
e_3\}$ möglich, solange genau eine davon rot ist.

Es bleibt noch zu zeigen, dass die Minimalität des Matchings in $G'$ äquivalent
zur Maximalität der geraden Menge in $G^*$ ist.

Da alle künstlichen Kanten in $M$ Gewicht 0 haben gilt:

$$ w(E'\backslash M) = w(E^*) - w(E^* \cap M) = w(E^*) - w(M) $$
$$ \implies w(E^*) = w(E'\backslash M) - w(M) $$

Die durch das Matching $M$ induzierte Menge $M^*$ ist möglicherweise leer.
Das wird später behandelt.

\textbf{Algorithmus zu Schritt 4}:\\
Wir wollen ein minimales perfektes Matching finden und kennen von vorher
bereits ein Algorithmus zum finden maximaler Matchings auf planaren Graphen.

Die Optimierungsrichtung können wir einfach umdrehen, indem wir die
Kostenfunktion verändern:
$$ w'(e) = W - w(e) $$

$W$ ist hierbei eine ausreichend große Konstante.

Die Perfektheit der Matchings können wir auch nocht erzwingen.
Bei perfekten Matchings gilt:
$$ w'(M) = \sum_{e \in M}{w'(e)} = \sum_{e \in M}{W - w(e)} =
\frac{n}{2}W - \sum_{e \in M}{w(e)} \geq \frac{n}{2}W - \frac{n}{2}w_{max}$$

$w_{max}$ ist hierbei das maximale originale Kantengewicht.
Für ein kleineres, nicht perfektes Matching $M^-$ gilt das Gegenteil:
$$ w'(M^-) \leq (\frac{n}{2} - 1) * (W - w_{min}) $$

Wir können also erzwingen, dass alle perfekten Matchings größer sind als alle
nicht perfekten Matchings, indem wir $W$ folgendermaßen wählen:
$$ W > \frac{n}{2} * (w_{max} - w_{min}) + w_{min} $$

Danach gibt uns der Algorithmus zwangsweise ein perfektes minimales Matching.

\textbf{Algorithmus zu Schritt 5}:\\
Es kann passieren, dass $M$ eine leere Menge in $M*$ induziert.
Das passiert dann, wenn ein paar aus zwei Kanten aus $\{e_1, e_2, e_3\}$ für
jeden Knoten ein negatives Gewicht hat und der Algorithmus demnach für jeden
Knoten die linke Variante aus Abbildung \ref{fig:shih2} auswählt.

Das können wir lösen, indem wir die Konstruktion aus Abbildung \ref{fig:shih1}
nochmal erweitern.
Siehe dazu Abbildung \ref{fig:shih3}.

\imageFigure{}{shih3.png}{.7}{shih3}

Diese Erweiterung wird für \textit{genau einen} Knoten $v$ verwendet.
Die zwei neuen Kanten erhalten wieder Gewicht 0.

Mit der neuen Konstruktion ist für $v$ nur noch die rechte Variante möglich, da
im neuen Matching die Kanten $M_v$ $\{w', u'\}$ und $\{w'', u''\}$ enthalten
sein müssen.
Dadurch ist zwangsweise genau eine der $e$-Kanten enthalten und es werden die
zwei anderen in $G^*$ induziert.
Damit ist die durch $M_v$ in $G^*$ induzierte Kantenmenge gerade.
Auch hier gilt:
$$ M_v \text{ perfekt und minimal} \iff M_v^* \text{ gerade und maximal} $$

Wir können mit dem Algorithmus von vorher das Matching in $\mathcal{O}(n*\log
n)$ $M$ zu $M_v$ erweitern, indem wir für $v$ erst $w'$ und dann $w''$
hinzufügen.

Wenn wir das ganze jetzt für jedes $v$ in $V^*$ ausprobieren und das kleinste
der minimale Matchings $M_v$ aussuchen, finden somit in $\mathcal{O}(n^2*\log
n)$ eine nicht-leere maximale gerade Menge in $V^*$.


TODO: Schnellere Version mit PST % TODO

TODO: Via-Minimierung % TODO



%%%%%%%%%%%%%%%%%%%%%%%%%%%%%%%%%%%%%%%%%%%%%%%%%%%%%%%%%%%%%%%%%%%%%%%%%%%%%%%

\clearpage
\section{\textit{st}-planare Graphen}

TODO % TODO

\subsection{Max-Flow}

TODO % TODO

%%%%%%%%%%%%%%%%%%%%%%%%%%%%%%%%%%%%%%%%%%%%%%%%%%%%%%%%%%%%%%%%%%%%%%%%%%%%%%%

\clearpage
\section{Menger}
\textbf{Menger-Problem}\\
Sei $G$ ein Graph und $s, t$ Knoten in $G$.
Finde eine maximale Anzahl an kanten- bzw. intern knotendisjunkten
$s$-$t$-Wegen.

\subsection{Kantendisjunkte Version}
Folgender Algorithmus läuft in Linearzeit auf einem planaren Graphen $G$.
Sei $G$ so planar eingebettet, dass $t$ an der äußeren Facette liegt.

\begin{enumerate}
    \item Ersetze in $G$ alle ungerichteten Kante $\{u, v\}$ durch die
        beiden gerichteten Kanten $(u,v)$ und $(v, u)$.
        G ist nun ein gerichteter Graph.
    \item Berechne eine Menge \textit{geeigneter} einfacher kantendisjunkter
        Kreise und betrachte den Graphen $G_C$, der entsteht, wenn man alle
        Kreiskanten umdreht.
    \item Berechne $s$-$t$-Wege auf $G_C$.
    \item Konstruiere aus den $s$-$t$-Wegen in $G_C$ eine gleiche Anzahl
        $s$-$t$-Wege in $G$.
        Dabei wird von den in Schritt 1 hinzugefügten Kanten jeweils nur eine
        benutzt, sodass die Wegemenge auch für den originalen ungerichteten
        Graphen gültig ist.
\end{enumerate}

\textbf{Ausführung von Schritt 1}:\\
\textbf{Lemma 7.2.} Man kann jede Wegemenge $P_d$ aus kantendisjunkten
$s$-$t$-Wegen so zu einer Menge $P$ umbauen, dass kein in Schritt 1 eingefügtes
Kantenpaar doppelt benutzt wird.

\textbf{Beweis.}\\
Wenn das Kantenpaar von einem einzigen Weg $p$ doppelt benutzt wird, dann
stellt das einen unnötigen Umweg da und $p$ kann durch einen einfachen Weg
ersetzt werden, der das Kantenpaar gar nicht benutzt.

Wenn stattdessen das Kantenpaar von zwei unterschiedlichen Wegen $p_i$ und
$p_j$ benutzt wird, dann können wir die Wege aufteilen in einen Teil vor dem
Kantenpaar und einen Teil danach.
Fügen wir nun beide Nachfolgeteile jeweils an den Vorgängerteil exkl. der Kante
aus dem Kantenpaar des anderen Weges an, erhalten wir zwei $s$-$t$-Wegen, die
das Kantenpaar nicht benutzen.

\textbf{Ausführung von Schritt 2}:\\
Wir konstruieren eine Menge einfacher kantendisjunkter Kreise $C$ auf $G$, die
auf $G_C$ folgende Eigenschaften induziert:

\begin{enumerate}
    \item $G_C$ enthält keinen Rechtskreis.
        Das Innere jedes Kreises liegt also links von den Kanten.
    \item Wir können eine Menge $P_C$ kantendisjunkter $s$-$t$-Wege auf $G_C$
        bilden und daraus eine Menge $P$ auf $G$ konstruieren, für die gilt:

        $$ P \text{ maximal} \iff P_C \text{ maximal} $$

        $P$ sind dabei alle Kanten aus $P_C$ mit folgenden Änderungen:
        \begin{itemize}
            \item Alle Kanten auf $P_C$ die vorher umgedreht wurden, also nicht
                so in $G$ vorkommen, werden nicht übernommen.
            \item Alle Kanten die vorher umgedreht wurden und \textit{nicht}
                auf $P_C$, werden in ihrer originalen Ausrichtung
                hinzugenommen.
                Das sind genau die Kanten, die auf $C$, aber nicht auf $P_C$
                liegen.
        \end{itemize}
\end{enumerate}

\imageFigure{}{menger1.png}{.75}{menger1}

Sei $dist(f)$ der Abstand einer Facette zur äußeren Facette $f_0$, also der
Abstand zwischen dem Dualknoten von $f$ und $f_0$.
Sei $l$ der maximale Abstand einer Facette zur äußeren.

Wir nehmen uns nun alle Kreise $C$ in $G$, die den Übergang zwischen zwei
Abständen markieren, also alle Kreise $c_i$, dessen innere Facetten Distanz
$\geq i$ und dessen äußere Facetten Distanz $< i$ haben.
Siehe Abbildung \ref{fig:menger2}.

\imageFigure{}{menger2.png}{.75}{menger2}

Bei jedem dieser Kreise in $C$ haben wir auf $G$ zwei Varianten: Den Linkskreis
und den Rechtskreis.
Wir nehmen alle Rechtskreise, drehen sie um und erhalten so $G_C$.

Jede Kreis in $G$ umschließt mindestens eine Facette.
Demnach muss auch jeder Kreis zu Teilen in $C$ vorkommen.
Da diese Teile umgedreht wurden, gibt es auch $G_C$ keine Rechtskreise mehr.
Eigenschaft 1 ist also erfüllt.

Für Eigenschaft 2 benutzt wir folgendes

\textbf{Lemma 7.4.} Sei $H$ ein gerichteter zusammenhängender Graph und $s, t$
zwei Knoten auf $H$.
$H$ besteht aus $k$ kantendisjunkten $s$-$t$-Wegen genau dann, wenn folgendes
gilt:
\begin{itemize}
    \item Außer für $s$ und $t$ ist Ausgangsgrad gleich Eingangsgrad.
        Es produziert also kein Knoten ein \enquote{Bottleneck}.
    \item $k = d^{out}(s) - d^{in}(s) = d^{in}(t) - d^{out}(t)$.
        $k$ ist also genau so groß wie die Anzahl Kanten, die von $s$
        \enquote{produziert} und von $t$ \enquote{konsumiert} werden.
\end{itemize}

Beweis im Skript. Hier weggelassen.

Wenn wir von $P_C$ nach $P$ wechseln ändern sich für die beteiligten Knoten
der Ausgangsgrad und der Eingangsgrad um den gleichen Betrag.

TODO: Randfälle, für die sich die Grade anders ändern. % TODO

$\implies$ Aus $k$ kantendisjunkten $s$-$t$-Wegen auf $G_C$ lassen sich $k$
kantendisjunkte $s$-$t$-Wege auf $G$ konstruieren.

\textbf{Ausführung von Schritt 3}:\\
Algorithmus: Siehe Abbildung \ref{fig:menger3}.
\imageFigure{}{menger3.png}{1}{menger3}

TODO: Schritt 4 % TODO

\subsection{Knotendisjunkte Version}
\begin{enumerate}
    \item Ersetze in $G$ alle ungerichteten Kante $\{u, v\}$ durch die
        beiden gerichteten Kanten $(u,v)$ und $(v, u)$, außer sie sind inzident
        zu $s$ oder $t$.
        Für $s$ nur die ausgehenden Kanten.
        Für $t$ nur die eingehenden Kanten.
        G ist nun ein gerichteter Graph.
        Eine Menge knotendisjunkter $s$-$t$-Wege auf $G$ induzieren direkt eine
        gleich große Menge knotendisjunkter $s$-$t$-Wege auf dem originalen
        Graphen.
    \item Finde $s$-$t$-Wege auf $G$ mittels Right-First-Tiefensuche.
        Wende bei Konflikten in besuchten Knoten nachfolgende Behandlung an.
\end{enumerate}

Bei \textbf{Konflikt von links}:\\
Lösche aktuell benutzte Kante und gehe einen Schritt zurück.

Bei \textbf{Konflikt von rechts}:\\
Teile den schon vorhandenen Pfad $p$ in den Teil $p_-$ vor $v$ und den Teil
$p_+$ danach.
Vertausche $p_-$ und den aktuellen Suchpfad, sodass der aktuelle Suchpfad mit
$p_+$ einen Weg ergibt.
$p_-$ ist nun der aktuelle Suchpfad und hat demnach gerade eine Kollision von
links ausgelöst, die wie oben beschrieben behandelt wird.

Die Konfliktbehandlung von rechts funktioniert nicht, wenn $p_+$ Teil des
aktuellen Suchpfads ist, also noch nicht Teil eines abgeschlossenen Weges.
Das passiert nur dann, wenn wir durch einen Rechtskreis an einen früheren
Knoten im gleichen Suchpfad gelangen.
Da wir Right-First-Auswahl verwenden, passiert das aber nur, wenn wir vorher
den umgekehrten Linkskreis gelaufen sind.
Wir können dieses Problem unterbinden, indem wir Knoten als \textit{Teil des
aktuellen Suchpfads} markieren und bei dem Konflikt dann stattdessen die
Behandlung für Links-Konflikte verwenden.

Für die Zugehörigkeit zum aktuelle Suchpfad geben wir jedem Knoten einen
Timestamp, sobald er zum Kopf des Suchpfades wird.
Der globale Timestamp-Counter wird erhöht, wenn wir einen neuen Weg von $s$ aus
starten oder Wege bei der Konfliktbehandlung von rechts umsortieren.
Ein vorheriger Knoten ist dann genau dann noch Teil des aktuellen Suchswegs,
wenn der Timestamp des vorherigen Knoten dem Timestamp des Suchkopfes gleicht.

\imageFigure{}{menger4.png}{1}{menger4}

%%%%%%%%%%%%%%%%%%%%%%%%%%%%%%%%%%%%%%%%%%%%%%%%%%%%%%%%%%%%%%%%%%%%%%%%%%%%%%%

\clearpage
\section{Okamura-Seymour}
\textbf{Kantendisjunktes Wegpackungsproblem}\\
Sei $G$ ein Graph und $\{s_1, t_1\}, \dots, \{s_k, t_k\}$ Paare von $s$- und
$t$-Knoten.
Finde paarweise kantendisjunkte $s_i$-$t_i$-Wege $p_i$.

Das Problem ist auch auf planaren Graphen $\mathcal{NP}$-vollständig.

Die \textit{Kapazität} $cap(X)$ eines Knoten-Teilmenge $X$ ist die Anzahl Kanten, die
über ihre Grenze verläuft.
Das ist auch die Größe des von ihr induzierten Schnittes.

Die \textit{Dichte} $dens(X)$ einer Knoten-Teilmenge $X$ ist die Anzahl
$s$-$t$-Paare, von denen genau ein Teil in $X$ liegt.

Offensichtlich müssen zur Erfüllung des Wegpackungsproblems alle möglichen
Knoten-Teilmengen von $G$ eine Kapazität $\geq$ ihrer Dichte haben.

Dazu sei $fcap(X)$ die freie Kapazität von $X$ mit
$$ fcap(X) = cap(X) - dens(X) $$

Falls $fcap(X) = 0$, dann ist $X$ \textit{saturiert}.\\
Falls $fcap(X) < 0$, dann ist $X$ \textit{übersaturiert}.

Die \textit{Kapazitätsbedingung} fordert, dass $fcap(X) \geq 0$ für
alle Teilmengen $X$ gilt.
Sie ist eine notwendige Bedingung für die Lösbarkeit des Wegpackungsproblems.

Die \textit{Geradheitsbedingung / Euler-Bedingung} fordert, dass $fcap(X)$
gerade ist für alle Teilmengen $X$.

\textbf{Okamura \& Seymour-Problem}\\
Gegeben sei ein planarer Graph $G$ und eine Menge $D$ von $k$ $s$-$t$-Paare,
die alle auf dem Rand derselben Facette liegen.
O.B.d.A sei eine Einbettung von $G$ gegeben, bei der diese Facette die äußere
Facette ist.
Außerdem sei die Geradheitsbedingung erfüllt.
Finde paarweise knotendisjunkte $s_i$-$t_i$-Wege $p_i$.

\textbf{Lemma 8.2.}
Die Geradheitsbedingung ist genau dann erfüllt, wenn $fcap(\{v\})$ gerade für
alle Knoten $v$.

Beweis im Skript. Hier ausgelassen.

\textbf{Satz von Okamura \& Seymour}\\
Ist zusätzlich noch die Kapazitätsbedingung erfüllt, dann ist das Okamura \&
Seymour-Problem lösbar.

\subsection{Algorithmus}

TODO % TODO

\end{document}
