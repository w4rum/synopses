\documentclass[10pt,a4paper]{article}
\author{Tim Schmidt}
\title{Algorithmen für planare Graphen}

\usepackage[utf8]{inputenc}
\usepackage{amsmath}
\usepackage{amssymb}
\usepackage{hyperref}
\usepackage{xspace}
\usepackage[german]{babel}
\usepackage[autostyle]{csquotes}
\usepackage[a4paper, total={6in, 8in}]{geometry}
\usepackage{siunitx}
\usepackage{tikz}
\usepackage{subfig}
\usetikzlibrary{shapes, arrows}
\usepackage[activate={true,nocompatibility},final,tracking=true,kerning=true,spacing=true,factor=1100,stretch=10,shrink=10]{microtype}
% activate={true,nocompatibility} - activate protrusion and expansion
% final - enable microtype; use "draft" to disable
% tracking=true, kerning=true, spacing=true - activate these techniques
% factor=1100 - add 10% to the protrusion amount (default is 1000)
% stretch=10, shrink=10 - reduce stretchability/shrinkability (default is 20/20)
\microtypecontext{spacing=nonfrench}

\setlength{\parindent}{0cm}
\setlength{\parskip}{0.2cm}

\graphicspath{{images/}}

% maxwidth parameter for \includegraphics
% see https://tex.stackexchange.com/a/86355
\makeatletter
\def\maxwidth#1{\ifdim\Gin@nat@width>#1 #1\else\Gin@nat@width\fi}
\makeatother

\newcommand{\imageFigure}[4]{%
    \begin{figure}[h]%
        \centering%
        {%
            \setlength{\fboxsep}{1pt}%
            \setlength{\fboxrule}{1pt}%
            %\fbox{
            \includegraphics[width=\maxwidth{#3\textwidth}]{#2}%}%
        }%
        \caption{#1}%
        \label{fig:#4}%
    \end{figure}%
}

\newcommand{\imageFigureMult}[6]{%
    \imageFigureMultS{#1}{#2}{#3}{#4}{#5}{#6}{0.45}%
}

\newcommand{\imageFigureMultS}[7]{%
    \begin{figure}[h]%
        \centering
        \subfloat[#1]{{\includegraphics[width=\maxwidth{#7\textwidth}]{#2} }}%
        \qquad
        \subfloat[#3]{{\includegraphics[width=\maxwidth{#7\textwidth}]{#4} }}%
        \caption{#5}%
        \label{fig:#6}%
    \end{figure}
}

\newcommand{\Kf}{$K_5$}
\newcommand{\Kdd}{$K_{3,3}$}

\begin{document}
	\pagenumbering{Roman}
	{\let\newpage\relax\maketitle}
	\tableofcontents
	\newpage
	\pagenumbering{arabic}
	\setcounter{page}{1}

%%%%%%%%%%%%%%%%%%%%%%%%%%%%%%%%%%%%%%%%%%%%%%%%%%%%%%%%%%%%%%%%%%%%%%%%%%%%%%%

\section{Grundlagen}
Planare Graphen sind Graphen, die sich so in die Ebene einbetten lassen, dass
sich keine Kanten kreuzen.
Eine Einbettung ist dabei eine Zuweisung von Knoten auf Punkte und Kanten auf
Kurven zwischen den inzidenten Knoten im 2D-Raum.

Es gibt Graphen, die nicht planar sind.
Darunter sind der vollständige Graph mit fünf Knoten \Kf und der vollständig
bipartite Graph mit sechs Knoten \Kdd:

\imageFigureMult{\Kf}{k5.png}{\Kdd}{k33.png}{Nicht-planare Graphen}{nonplanar}

\subsection{Allgemeine Aussagen über Knoten- und Kantenmenge}
Über planare Graphen lassen sich folgende allgemeine Aussagen machen:
\begin{itemize}
    \item In einem \textit{zusammenhängenden} planaren Graph erfüllt die Anzahl
        der Knoten $n$, Kanten $m$ und Facetten $f$ den Satz von Euler:
        $$n-m+f=2$$
        Dies lässt sich relativ leicht durch Induktion zeigen:

        In einem einfach zusammenhängenden Pfad gibt es $n$ Knoten, $n-1$
        Kanten und genau eine Facette, nämlich die äußere, also gilt
        $$n-(n-1)+1=2$$

        Wird nun ein Knoten hinzugefügt und mit einer Kante an den Graphen
        angehangen, erhöhen sich $n$ und $m$ jeweils um eins.
        Dabei verändert sich die Summe nicht.
        $$(n+1)-(n-1+1)+1=2$$

        Wird eine Kante zwischen zwei bestehenden Knoten eingefügt, wird damit
        eine Facette durchschnitten.
        Somit erhöht sich $m$ und $f$ um eins und die Summe bleibt weiterhin
        gleich.
        $$n-(n-1+1)+(1+1)=2$$

        \imageFigureMult{$n=4, m=3, f=1$}{line.png}
            {$n=4, m=4, f=2$}{line2.png}
            {Bzgl. Satz von Euler}{eulergraphs}
    \item Ein planarer, \textit{einfacher} Graph mit mindestens 3 Knoten hat
        höchstens $3n-6$ Kanten.
        Wenn er keine Dreiecke hat, sogar nur höchstens $2n-4$ Kanten.

        Dies ist extrem wichtig für die Laufzeiten einiger Algorithmen, da für
        planare Graphen somit $m \in \mathcal{O}(n)$ gilt, während für
        allgemeine Graphen nur $m \in \mathcal{O}(n^2)$ gilt.

        Das lässt sich aus dem Satz von Euler schließen:

        In einem planaren Graph mit maximaler Anzahl von Kanten ist jede
        Facette ein Dreieck (Beweis später).
        Demnach grenzt jede Facette an drei Kanten an.
        Da jede Kante gleichzeitig an zwei Facetten angrenzt gilt:
        $$3f=2m$$
        Eingesetzt in den Satz von Euler:
        \begin{align*}
            n-m+\frac{2}{3}m&=2\\
            n-\frac{1}{3}m&=2\\
            m&=3n-6
        \end{align*}

        Dies gilt für maximale planare Graphen.
        Im Allgemeinen also
        $$m \leq 3n-6$$

        In Graphen ohne Dreiecke ist der Beweis analog mit $4f\leq2m$


    \item Jeder planare Graph enthält einen Knoten mit Grad $\leq 5$

        Sei $n_i$ die Anzahl Knoten mit Grad $i$ und $d_{max}$ der Maximalgrad.
        Dann gilt
        $$n=\sum_{i=0}^{d_{max}}{n_i}$$
        und
        $$2m=\sum_{i=0}^{d_{max}}{i*n_i}$$
        Eingesetzt in die Kantenbegrenzung:
        \begin{align*}
            m &\leq 3n-6\\
            2m &\leq 6n-12\\
            2m+12 &\leq 6n\\
            \sum_{i=0}^{d_{max}}{i*n_i} + 12 &\leq 6*\sum_{i=0}^{d_{max}}{n_i}\\
        \end{align*}
        \begin{align*}
            0*n_0+1*n_1+2*n_2+3*n_3+&4*n_4+5*n_5+6*n_6+7*n_7+\cdots + 12 \leq\\
            6*n_0+6*n_1+6*n_2+6*n_3+&6*n_4+6*n_5+6*n_6+6*n_7\cdots
        \end{align*}
        Subtrahiert man $n_0$ bis $n_6$ nun aus der oberen Zeile raus erhält man
        $$1*n_7+2*n_8+\cdots + 12 ~~~\leq~~~ \underbrace{6*n_0+5*n_1+4*n_2+3*n_3+2*n_4+1*n_5}_{\geq 1\text{, also gibt es mind. einen Knoten in }n_0\text{ bis } n_5}\\$$
        %%
\end{itemize}
% Kanten bei keinen Dreiecken
% Satz von Euler
% Knoten mit Grad <= 5
% Färbbarkeit
\subsection{Triangulierung}
Ein \textit{triangulierter} planarer Graph ist ein Graph, in dem alle Facetten
Dreiecke sind.
Dies ist gleichzeitig ein \textit{maximal planarer} Graph, d.h. es kann keine
weitere Kante hinzugefügt werden, ohne die Planarität zu zerstören.
\begin{itemize}
    \item[\enquote{$\Longrightarrow$}]
        \begin{itemize}
            \item Angenommen Graph ist maximal planar aber nicht trianguliert
            \item Graph nicht trianguliert $\implies$ Es gibt Facette $f$, die
                kein Dreiecke ist
            \item Auf Facette $f$ gibt es Knoten $u, v$, die nicht verbunden
                sind
            \item Kante $\{u, v\}$ kann hinzugefügt werden
            \item Graph ist nicht maximal planar. Widerspruch.
        \end{itemize}
    \item[\enquote{$\Longleftarrow$}]
        \begin{itemize}
            \item Angenommen Graph ist trianguliert aber nicht maximal planar
            \item Es kann eine Kante eingefügt werden, die keine Kreuzung
                verursacht.
            \item Diese Kante muss also durch eine einzige Facette f verlaufen.
            \item Da f ein Dreieck ist, sind alle Knoten in f aber schon
                verbunden. Widerspruch.
        \end{itemize}
\end{itemize}

\subsection{Dualgraphen}
Zu jedem planaren, eingebetteten Graph $G$ gibt es einen \textit{Dualgraph}
$G^*$.
Dieser wird erzeugt, indem jede Facette in $G$ einen Knoten in $G^*$ bekommt
und für jede Kante in $G$ die beiden anliegenden Facetten in $G^*$ verbunden
werden.

\imageFigure{Dualgraph}{dual.png}{.5}{dual}

\textit{Bemerkungen}:
\begin{itemize}
    \item Der Dualgraph muss nicht einfach sein.
        Es treten Multikanten zwischen Dualknoten auf, wenn zwei Facetten in
        $G$ durch mehr als eine Kante verbunden sind.
        Dies ist in triangulierten Graphen offensichtlich nicht möglich.

        Außerdem können Schleifen entstehen, wenn beide Seiten einer Kante in
        $G$ zur gleichen Facette gehören.
        Solche Kanten werden auch \enquote{Brücken} genannt und verbinden zwei
        sonst unzusammenhängende Teilgraphen.
    \item Einige Graphen sind isometrisch zu ihrem Dualgraphen, d.h. man kann
        ihre Dualgraphen so einbetten, dass sie dem Ausgangsgraphen gleichen.
        Solche Graphen nennt man \textit{selbstdual}.
    \item Dualgraphen sind immer abhängig von einer konkreten Einbettung des
        Ausgangsgraphen.
        Ein Graph kann abhängig von der Einbettung verschiedene Dualgraphen
        haben.
\end{itemize}

%%%%%%%%%%%%%%%%%%%%%%%%%%%%%%%%%%%%%%%%%%%%%%%%%%%%%%%%%%%%%%%%%%%%%%%%%%%%%%%

\section{Satz von Kuratowski}
Wie bereits oben vorgestellt gibt es mindestens zwei Graphen, die nicht planar
sind:
Der \Kf~und der \Kdd.

Offensichtlich sind auch all die Graphen nicht-planar, die den \Kf~oder den
\Kdd~als Subgraph enthalten.
Ebenso werden diese beiden Graphen nicht dadurch planar, indem wir einen Knoten
mitten auf einer Kante einfügen.
Ein so entstandener Graph nennt man auch eine \enquote{Unterteilung}.
\imageFigure{Unterteilung von \Kf}{k5-unt.png}{.2}{k5-unt}

Tatsächlich reichen diese zwei Graphen und das Verständnis von Unterteilungen,
um die Menge der planaren Graphen vollständig zu charakterisieren.
% Aussage
\subsection{Kantenkontraktion}
Um eine Unterteilung des \Kf oder \Kdd festzustellen, brauchen wir eine
Technik, um Unterteilungen rückgängig machen zu können.
Diese nennt sich Kantenkontraktion.

Beim Kontrahieren einer Kante $\{u,v\}$ werden die beiden anliegenden Knoten $u$
und $v$ zu einem Knoten $vu$ zusammengelegt und die Kante $\{u,v\}$ gelöscht.
Die restlichen vorher an $u$ und $v$ anliegenden Kanten werden alle auf $vu$
übertragen.
Dadurch entstandene Multikanten werden gelöscht.

\imageFigure{Kontraktion von $\{u,v\}$}{kontr.png}{.7}{kontr}

Entfernt man aus Abbildung \ref{fig:k5-unt} den roten Knoten, indem man eine
der beiden anliegenden Kanten kontrahiert, erhält man wieder genau den \Kf.

Beachte das beim Rückgängigmachen von Unterteilungen immer einer der Knoten,
der an der Kontraktion beteiligt ist, Grad 2 hat.

\subsection{Minor, Topologischer Minor}
Sei $G$ ein Graph.
Wenn es möglich ist den Graphen $H$ per Kantenkontraktion aus einem Subgraphen
von $G$ zu erhalten, dann ist $H$ ein \textit{Minor} von $G$.
Ist dies sogar dann möglich, wenn man nur Kanten kontrahiert, an die ein Knoten
mit Grad 2 anliegt, dann ist $H$ sogar ein \textit{topologischer Minor}.

Da es erlaubt ist, nur einen Subgraphen von $G$ zu betrachten, darf man also
wie folgt vorgehen:
\begin{enumerate}
    \item Entferne beliebig viele Knoten und die anliegenden Kanten aus $G$.
    \item Kontrahiere beliebig viele Kanten in $G$ (beachte obige Beschränkung,
        falls nach topologischen Minoren gesucht wird).
    \item Falls dadurch \Kf~oder \Kdd~rauskommt ist $G$ nicht planar.
\end{enumerate}

\subsection{Satz von Kuratowski / Wagner}
Nach dem \textbf{Satz von Kuratowski} sind \textit{genau} die Graphen planar,
die weder \Kf~noch \Kdd~als topologischen Minor enthalten.

Der \textbf{Satz von Wagner} hingegen besagt, dass \textit{genau} die Graphen
planar sind, die weder \Kf~noch \Kdd~als Minor enthalten.
Diese Minoren müssen nicht zwangsweise topologische Minoren sein.

Folgende Anmerkungen:
\begin{enumerate}
    \item Die Menge der Minoren eines Graphen und die Menge der topologischen
        Minoren eines Graphen sind \textit{nicht} identisch.
    \item Trotzdem sind die Sätze von Kuratowski und Wagner äquivalent.
\end{enumerate}

Beispiel zu Punkt 1:\\
Im Petersengraph in Abbildung \ref{fig:petersen} erhält man \Kf~als Minor,
indem man alle roten Kanten kontrahiert.
Der \Kf~enthält 5 Knoten mit Grad 4.
Durch eine Unterteilung werden die Grade der ursprünglichen Knoten nicht
verändert.
Demnach müsste der Petersengraph auch 5 Knoten mit mindestens Grad 4 besitzen,
um \Kf~als topologischen Minor zu enthalten.
Dies ist nicht der Fall.

Allerdings enthält der Petersengraph den \Kdd~als topologischen Minor.
Die Sätze von Kuratowski und Wagner kommen hier also zum gleichen Ergebnis.

\imageFigure{Petersengraph}{petersen.png}{.7}{petersen}


\subsection{Beweis der Äquivalenz}
Nun ein Beweis zum obigen Punkt 2.

Wir zeigen zuerst, dass wenn \Kdd~als Minor vorkommt, er auch als topologischer
Minor vorkommt, indem wir folgendes Lemma beweisen:

\textbf{Lemma 1.} Wenn der gesuchte Graph $H$ Maximalgrad $\Delta \leq 3$
besitzt und ein Minor von $G$ ist, dann ist er auch ein topologischer Minor von
$G$.

$H$ ein Minor von $G$ und sei $G'$ der Subgraph von $G$, von dem aus wir durch
Kontraktionen auf $H$ kommen.
Angenommen es soll die Kante $\{u,v\}$ kontrahiert werden.
Da in dem Zielgraphen $H$ der Maximalgrad höchstens 3 hat, darf der
resultierende Knoten $uv$ auch maximal Grad 3 besitzen.\footnote{Es gibt
Kontraktionsfolgen, bei denen der Grad von Knoten nachträglich wieder abnimmt.
Keine Ahnung, warum das hier ignoriert werden darf. Es wird im Skript nicht
angesprochen.}

Angenommen $v$ und $u$ haben keine gemeinsamen Nachbarn\footnote{Die
Randfallbehandlung für den anderen Fall im Skript scheint keinen Sinn zu
ergeben, falls der gemeinsame Nachbar Grad 3 hat.}, dann gilt
$$ \delta(vu) = \delta(u) - 1 + \delta(v) - 1 $$

Dies kann aber nur dann höchstens 3 sein, wenn entweder $v$ oder $u$ einen Grad
von höchstens 2 hat.

\imageFigureMultS{$uv$ hat danach Rang 3}{eq-contr.png}{$uv$ hat hier auch Rang 3}{eq-contr2.png}{Kontraktion der Kante $\{u, v\}$}{eq-contr}{.3}

Da einer der Knoten also höchstens Grad 2 hat, darf diese Kontraktion auch
unter topologischer Minorenfindung durchgeführt werden.

$\implies$ Wenn \Kdd~Minor ist, ist er auch topologischer Minor.

Nun zeigen wir noch, dass wenn \Kf~Minor ist, entweder \Kf~oder
\Kdd~topologischer Minor ist.

Sei \Kf~Minor von $G$ und $G'$ der minimale Subgraph, aus dem \Kf~nur durch
Kontraktionen gewonnen wird.
Dann gibt es eine Partition von $G'$ in fünf Knotenmengen $\{V_1, \cdots,
V_5\}$.
Diese sind paarweise miteinander über eine Kante verbunden.
Da $G'$ minimal ist, gibt es keine Kreise innerhalb der Partitionen, weshalb
sie für sich genommen Bäume sind.

Wenn alle $V_i$ Unterteilungen von $K_{1,4}$ sind (wie $V_5$ in Abbildung
\ref{fig:minor-to-top}), dann ist $G'$ eine Unterteilung von \Kf~und hat
demnach \Kf~als topologischen Minor.

Wenn eine Partition keine Unterteilung von $K_{1,4}$ ist, dann muss sie zwei
Knoten mit Grad 3 haben und wir können wir in Abbildung \ref{fig:minor-to-top}b
einen \Kdd-Minor bauen.
Wie oben gezeigt ist dann \Kdd~auch ein topologischer Minor.

$\implies$ Wenn \Kf~ein Minor ist, dann ist \Kf~oder \Kdd~ein topologischer
Minor.

\imageFigureMult{}{minor-to-top1.png}{}{minor-to-top2.png}{}{minor-to-top}

Demnach sind also die Sätze von Kuratowski und Wagner äquivalent.

\subsection{Beweis des Satzes von Wagner}
Es ist einfacher den Satz von Wagner zu beweisen.
Da dieser mit dem Satz von Kuratowski äquivalent ist müssen wir letzteren
danach nicht mehr beweisen.

Dass ein Graph nicht planar sein kann, wenn er einen nicht-planaren Graphen als
Subgraphen oder Minor hat, ist offensichtlich.
Es bleibt also nur die andere Richtung zu beweisen.

Dazu definieren wir \textit{minorminimal nicht planare} Graphen.
Das sind Graphen, die selbst nicht planar sind, aber dessen Minoren alle planar
sind.

Ein solcher Graph hat Minimalgrad 3, da Knoten mit Grad $\leq 2$ kontrahiert
werden können, ohne die Planarität zu ändern, was zu einem nicht-planaren Minor
führen würde.

Offensichtlich ist auch, dass wenn ein Graph $G$ nicht planar ist aber nicht
selbst minorminimal nicht planar ist, einen minorminimal nicht planaren Minor
haben muss.
Das ist leicht zu realisieren wenn man bedenkt, dass man jeden Graphen planar
machen kann, wenn man nur genug Knoten entfernt oder Kanten kontrahiert.
Macht man das so lange, bis jede weitere Entfernung bzw. Kontraktion zu einem
planaren Graphen führen würde, hat man einen minorminimal nicht planaren
Graphen erreicht.

Im folgenden zeigen wir, dass \Kf~und \Kdd~die einzigen minorminimal nicht
planaren Graphen sind.
Daraus folgt dann direkt, dass entweder \Kf~oder \Kdd~enthalten sein muss,
damit ein Graph nicht planar ist.

Dieser Beweis ist ziemlich lang und wir brauchen dafür einiges an Werkzeug, das
wir vorher abarbeiten werden.
Den gesamten Beweis lang haben wir einen Graph $G$, der minorminimal nicht
planar ist und einen Graph $G'$, der entsteht, wenn wir zwei beliebige
\textit{benachbarte} Knoten $x$ und $y$ entfernen.

\textbf{Lemma 3.} $G'$ ist ein einfacher Kreis.

Um Lemma 3 zu beweisen brauchen wir zwei weitere Lemmata:

\textbf{Lemma 5.} $G'$ enthält keinen Theta-Graph (siehe Abbildung
\ref{fig:theta}).\\
\textbf{Lemma 6.} $G'$ enthält höchstens einen Knoten mit Grad $\leq 1$.

\imageFigure{Ein Theta-Graph ist eine beliebige Unterteilung des obigen
Multigraphs}{theta.png}{.3}{theta}

\textbf{Beweis zu Lemma 5}:\\
Folgende Feststellungen sind wichtig:
\begin{itemize}
    \item Nur Bäume und Wälder haben genau eine Facette.
        Bei Graphen mit mehreren Facetten bildet jeder Facettenrand einen
        Kreis.
    \item Sei $G$ planar und $C$ ein einfacher Kreis in $G$.
        Wenn alle Knoten auf $C$ nur durch $C$ verbunden sind, also in $G-E(C)$
        nicht verbunden sind, dann kann man $G$ so einbetten, dass $C$ die
        äußere Facette begrenzt. Siehe Abbildung \ref{fig:c-outer}.
    \item Der Rand $F$ einer Facette $f$, also alle an $f$ anliegenden Knoten
        und Kanten, beinhaltet keinen Theta-Graph. Siehe Abbildung
        \ref{fig:face-no-theta}.
\end{itemize}

\imageFigure{C begrenzt die äußere Facette}{c-outer.png}{.5}{c-outer}
\imageFigure{Facettenränder beinhalten keinen Theta-Graphen}{face-no-theta.png}{.25}{face-no-theta}

Da $G$ minorminimal nicht planar ist, muss $G'$ planar sein.
Wir können also eine planare Einbettung $\Gamma_{G'}$ von $G'$ erstellen.
Wir können sie sogar so konstruieren, dass alle zu $x$ oder $y$ benachbarten
Knoten auf dem Rand der gleichen Facette $f$ liegen.

Wenn wir jetzt den Facettenrand von $f$ betrachten, dann enthält dieser
nachobiger Beaobachtung keinen Theta-Graphen.
Angenommen $G'$ enthält einen Theta-Graph, dann ist $G'$ kein Baum oder Wald
und der Facettenrand von $f$ beinhaltet einen Kreis $C$.

Wir können die Einbettung $\Gamma_{G'}$ so konstruieren, dass auch die
entfernten Knoten $x$ und $y$ im Inneren des Kreises $C$ liegen.\footnote{Das
Skript benutzt hier die Notation $G/xy$, die nicht weiter erklärt wird.
Möglicherweise ist $xy$, also die Kontraktion der Kante $\{x, y\}$ gemeint.}

Da $G'$ einen Theta-Graphen enthält, muss sich midnestens noch eine weitere
Kante $e$ außerhalb des Facettenrandes, also in $E(G') \backslash E(F)$,
befinden.
Damit liegt diese Kante auch außerhalb des Kreises $C$, also in $ext(C)$.

Da $G$ minorminimal nicht planar und $ext(C)$ nicht leer ist, muss $G-ext(C)$
planar sein.
Da auch hier innerhalb von $C$ kein Theta-Graph besteht und demnach keine zwei
Knoten auf $C$ durch eine weitere Kante innerhalb von $C$ verbunden sind, kann
man wie in der zweiten obigen Beobachtung $G-ext(C)$ so einbetten, dass $C$
die äußere Facette begrenzt.\footnote{Aber warum sollte $C$ auf $G-ext(C)$
keinen Theta-Graphen enthalten? Diese Folgerung basierte auf $G'$. Hier ist $x$
und $y$ ja noch enthalten, also sollte es doch auch möglich sein, dass $C$
nicht Teil eines Facettenrandes ist.}

Wenn wir jetzt die Einbettungen $\Gamma_{G'}$ und $\Gamma_{G-ext(C)}$ jeweils
bei $C$ teilen und dann kombinieren, haben wir eine planare Einbettung von $G$.
Widerspruch.

$\implies$ $G'$ enthält keinen Theta-Graphen. (Lemma 5)

\textbf{Beweis zu Lemma 6}:\\
Zu beweisen ist, dass $G'$ höchstens einen Knoten mit Grad $\leq 1$ besitzt.

Knoten von Grad 0 sind nicht möglich:
$G$ hat Minimalgrad 3.
Durch entfernen von $x$ und $y$ kann der Grad eines Knotens höchstens auf 1
sinken, aber nicht auf 0.

Angenommen es gäbe zwei Knoten $v$ und $u$ in $G'$ mit Grad 1.
Dann müssten in $G$ beide adjazent zu $x$ und $y$ sein.
Da die restlichen Knoten in $G$ auch Minimalgrad 3 haben, sind $v$ und $u$ noch
über einen weiteren Weg $p$ verbunden.
Dadurch enthält $G$ einen Theta-Graph in $\{x,y,u,v\} \cup p$.

In diesem Fall darf es keine einzige Kante $\{i, j\}$ geben, die nicht an $\{x,
y, u, v\}$ inzident ist.
Ansonsten würde $G - i - j$ einen Theta-Graphen enthalten, was nach Lemma 5
nicht möglich ist.

\imageFigure{Es darf keine unabhängige Kante geben.}{xuvy1.png}{.3}{xuvy1}

Damit gibt es nur noch weniger Möglichkeiten für das Aussehen von $G$:

\imageFigure{
    $u$ und $v$ sind verbunden.
    Es gibt keine weiteren Knoten.
}{xuvy2.png}{.3}{xuvy2}

\imageFigure{
    Es gibt den weiteren Knoten $z$.
    $u$ und $v$ können nicht verbunden sein, da sie in $G$ höchstens Grad 3
    haben dürfen.
}{xuvy3.png}{.9}{xuvy3}

\imageFigure{
    Es gibt zwei weiteren Knoten $z$ und $z'$.
    Der rechte Graph ist eine planere Einbettung des linken Graphs.
    $u$ und $v$ können nicht verbunden sein, da sie in $G$ höchstens Grad 3
    haben dürfen.
    $z$ und $z'$ dürfen nicht verbunden sein, da sie eine unabhängige Kante
    bilden würden.
    Einen zusätzlichen Knoten $z''$ kann man nicht einfügen, da dieser
    mindestens Grad 3 haben müsste.
    Zu $v$ und $u$ dürfte er wegen des Maximalgrad von $v$ und $u$ nicht
    verbunden sein und zu $z$ und $z'$ nicht, weil es eine unabhängige Kante
    wäre.
}{xuvy4.png}{.7}{xuvy4}

Siehe Abbildungen \ref{fig:xuvy2}, \ref{fig:xuvy3} und \ref{fig:xuvy4}.
Demnach wäre $G$ auf jeden Fall planar, was einen Widerspruch darstellen würde.

$\implies$ $G'$ hat maximal einen Knoten mit Grad $\leq$ 1. (Lemma 6)

\clearpage
\textbf{Beweis zu Lemma 3}:\\
Wir beweisen nun, dass $G'$ ein einfacher Kreis ist.
Dazu nehmen wir an, dass $G'$ \textit{kein} einfacher Kreis ist und
schlussfolgern daraus den Widerspruch, dass $G$ planar ist.

Dazu definieren wir erst einmal die \textit{Blockzerlegung}.
Zwei Kanten $e_1$ und $e_2$ sind äquivalent im Sinne der Blockzerlegung, wenn
sie entweder gleich sind oder es einen einfachen Kreis gibt, der beide enthält.

Die so induzierte Äquivalenzklassenzerlegung sieht man bspw. in Abbildung
\ref{fig:block1}.
Jede Kante ist dabei in genau einem Block und es gibt möglicherweise Knoten,
die in mehreren Blöcken enthalten sind.
Diese nennt man Seperatorknoten, hier rot dargestellt.

\imageFigure{Blockzerlegung}{block1.png}{.7}{block1}

Nach Lemma 5 enthält $G'$ keinen Theta-Graphen, weshalb alle Blöcke entweder
einzelne Kanten oder einfache Kreise sind.
Auch ist es nicht möglich, dass alle Blöcke in einem Kreis angeordnet sind,
weil sie sonst zu einem Block zusammenfallen würden.
Es muss also \textit{Endblöcke} geben.

Nach Lemma 6 gibt es maximal einen Knoten mit Grad $\leq 1$.
Da $G'$ nach Vorraussetzung kein einfacher Kreis ist und es nach Lemma 6 auch
keine einzelne Kante ist, muss es mindestens zwei Blöcke und demnach auch zwei
Endblöcke geben.
Maximal einer dieser Endblöcke kann eine einfache Kante sein.
Der anderen muss ein Kreis sein.

\imageFigure{Blockzerlegung von $G'$}{block2.png}{.7}{block2}

Diesen Kreis nennen wir jetzt $C$ und den Separatorknoten darauf $v$.
Da alle anderen Knoten auf $C - v$ in $G'$ Grad 2 haben, aber $G$ Minimalgrad 3
hat, müssen diese Knoten zu $x$ oder $y$ adjazent sein.
Da ein Kreis mindestens Länge 3 hat, gibt es mindestens zwei andere Knoten auf
$C - v$.
Diese bilden wie in Abbildung \ref{fig:prisma1} zu sehen ist einen Theta-Graph
mit $\{x, y, v\}$.

\imageFigure{Alle anderen Knoten auf $C-v$ müssen zu $x$ oder $y$ adjazent
sein.
Alternative Kanten sind in Grau dargestellt.
}{prisma1.png}{.5}{prisma1}

Nach Lemma 5 gibt es keine unabhängige Kante, also keine Kante außerhalb von
$\{x, y\} \cup C$, da ihr Entfernen aus $G$ ein $G'$ mit Theta-Graph
produzieren würde.
Isolierte Knoten sind nach Lemma 6 auch unmöglich, weshalb es höchstens einen
weiteren Knoten $u$ geben kann.
Sieh Abbildung \ref{fig:prisma2}.

\imageFigure{Es gibt höchstens $u$ als zusätzlichen
Knoten.}{prisma2.png}{.4}{prisma2}

Da aber auch $G-u-v$ keinen Theta-Graph enthalten darf, muss $C$ Länge 3 haben.
Siehe Abbildung \ref{fig:prisma3}a.

Der resultierende Graph mit $|C| = 3$ ist aber der sogenannte Prisma-Graph.
Dieser ist planar. Widerspruch.

\imageFigureMultS{$C$ kann nicht Länge $> 3$ haben.}{prisma3.png}
{\textit{Prisma}-Graph}{prisma3.png}{}{prisma3}{.35}

$\implies$ $G'$ ist ein einfacher Kreis.

Nun können wir uns in Richtung des \textbf{Satzes von Wagner} bewegen.
Aus Lemma 3 folgern wir, dass $G$ aus einem Kreis $C$, zwei Knoten $x$ und $y$,
der Kante $\{x,y\}$ und Kanten zwischen $C$ und $x$ und $y$ besteht.

Vorher beweisen noch schnell ein kleines Lemma:

\textbf{Lemma 4}. Für zwei benachbarte Knoten $u$ und $v$ auf $C$ gilt:
Wenn $u$ zu exakt einem von $\{x, y\}$ verbunden, ist, dann ist $v$ nicht zu
diesem Knoten aus $\{x, y\}$ verbunden.
Formaler:
$$[~\{u,x\} \in E \land \{u,y\} \notin E~] \implies \{v,x\} \notin E$$

Sei o.B.d.A. und nur zur Veranschaulichung angenommen, dass $G$ aussieht wie in
Abbildung \ref{fig:lem4}.
Würde es die Kante $\{v, x\}$ geben, dann könnte sie entfernt werden, um einen
planaren Graphen aus $G$ zu erzeugen, da $G$ minorminimal nicht planar ist.

Wir können diese Kante aber danach wieder planar einfügen:
Knoten $u$ ist nur durch zwei Kanten an $C$ angeschlossen und durch eine Kante
an $x$.
Das bedeutet, dass unabhängig von der Einbettung die Kanten $\{u, v\}$ und
$\{u, x\}$ direkt aufeinander folgen.
Die Kante $\{v, x\}$ produziert also keine Kreuzung.
Damit ist $G$ planar einbettbar. Widerspruch.

$\implies$ Lemma 4 bewiesen.

\imageFigure{Die Kante $\{v, x\}$ darf es hier nicht
geben.}{lem4.png}{.4}{lem4}

Da $G$ Minimalgrad 3 hat, muss jeder Knoten auf $C$ mit mindestens einem von
$\{x, y\}$ verbunden sein.
Nach Lemma 4 sind dann entweder alle Knoten auf $C$ mit $x$ \textit{und} $y$
verbunden, oder sie sind abwechselnd mit nur einem davon verbunden.
Ersteres klingt schon sehr nach \Kf~und letzteres klingt nach der Zweifärbung,
die wir in \Kdd~haben.

In einem letzten Schritt begrenzen wir noch die Größe von $C$.
Angenommen $C$ hat Größe $\geq 5$.
Siehe Abbildung \ref{fig:csize}a.
Entfernt man zwei beliebige auf $C$ benachbarte Kanten $i$ und $j$, dann
verbleiben die Kanten $x, y, v_1, v_2, v_3$, wobei $v_2$ zwischen $v_1$ und
$v_3$ auf $C$ liegt.

\imageFigureMultS{$|C| \geq 5$}{csize1.png}
{$|C| = 4$ und alle Knoten sind mit $x$ und $y$ verbunden.}{csize2.png}
{Hinweis: Es sind nicht alle Kanten eingezeichnet.}{csize}{.2}

Da $v_2$ zu $v_1$ und $v_3$, aber auch zu einem von $\{x, y\}$ adjazent sein
muss\footnote{Warum?}, hat $v_2$ Grad 3, womit $G - i - j$ kein einfacher Kreis
ist. Widerspruch.

$\implies$ $C$ hat Größe 3 oder 4.

Falls $C$ Größe 4 hat und alle Knoten mit $x$ \textit{und} $y$ verbunden sind
folgt der gleiche Widerspruch.
Siehe Abbildung \ref{fig:csize}b.

$\implies$ Entweder $C$ hat Größe 3 oder $C$ hat Größe 4 und die Knoten sind
abwechselnd mit $x$ und $y$ verbunden.

Die abwechselnde Verbundenheit mit $x$ und $y$ ist auf einem Kreis der Größe 3
offensichtlich nicht möglich, weshalb im Falle $|C|=3$ alle Knoten mit $x$ und
$y$ verbunden sind.

Diese zwei Möglichkeiten ergeben genau den \Kf~und den \Kdd.
Siehe Abbildung \ref{fig:proof-k5}.

$\implies$ \Kf~und \Kdd~sind die einzigen minorminimal nicht planaren Graphen.

$\implies$ Alle nicht planaren Graphen beinhalten \Kf~oder \Kdd~als Minor.

\imageFigureMultS{\Kf}{proof-k5.png}{\Kdd}{proof-k33.png}{}{proof-k5}{.3}

%%%%%%%%%%%%%%%%%%%%%%%%%%%%%%%%%%%%%%%%%%%%%%%%%%%%%%%%%%%%%%%%%%%%%%%%%%%%%%%

\section{Färbung}
\subsection{4-Färbbarkeit}
\subsection{5-Listenfärbbarkeit}
\subsection{4-Listenfärbbarkeit}

% Verbindung Maximalgrad <-> Chromatische Zahl

%%%%%%%%%%%%%%%%%%%%%%%%%%%%%%%%%%%%%%%%%%%%%%%%%%%%%%%%%%%%%%%%%%%%%%%%%%%%%%%

\section{Kleinkram}
\subsection{Außenplanare Graphen}
\subsection{Petersengraph}
\subsection{Adjazenztest in $\mathcal{O}(1)$}
\subsection{Core-Zerlegung}
\subsection{Entartetheit}
\subsection{Dreiecke zählen}
\subsection{MST}

%%%%%%%%%%%%%%%%%%%%%%%%%%%%%%%%%%%%%%%%%%%%%%%%%%%%%%%%%%%%%%%%%%%%%%%%%%%%%%%

\section{Planar-Separator-Theorem}

%%%%%%%%%%%%%%%%%%%%%%%%%%%%%%%%%%%%%%%%%%%%%%%%%%%%%%%%%%%%%%%%%%%%%%%%%%%%%%%

\section{Planaritätstest mit PQ-Baum}

%%%%%%%%%%%%%%%%%%%%%%%%%%%%%%%%%%%%%%%%%%%%%%%%%%%%%%%%%%%%%%%%%%%%%%%%%%%%%%%

\section{Triangulierung}

%%%%%%%%%%%%%%%%%%%%%%%%%%%%%%%%%%%%%%%%%%%%%%%%%%%%%%%%%%%%%%%%%%%%%%%%%%%%%%%

\section{Matchings}
\subsection{Erhöhende Wege}

%%%%%%%%%%%%%%%%%%%%%%%%%%%%%%%%%%%%%%%%%%%%%%%%%%%%%%%%%%%%%%%%%%%%%%%%%%%%%%%

\section{Mixed-Max-Cut und Via-Minimierung}

%%%%%%%%%%%%%%%%%%%%%%%%%%%%%%%%%%%%%%%%%%%%%%%%%%%%%%%%%%%%%%%%%%%%%%%%%%%%%%%

\section{\textit{st}-planare Graphen}
\subsection{Max-Flow}

%%%%%%%%%%%%%%%%%%%%%%%%%%%%%%%%%%%%%%%%%%%%%%%%%%%%%%%%%%%%%%%%%%%%%%%%%%%%%%%

\section{Menger}


\end{document}
