\documentclass[10pt,a4paper]{article}
\author{Tim Schmidt}
\title{Algorithms for graph visualization}

\usepackage[utf8]{inputenc}
\usepackage{amsmath}
\usepackage{amssymb}
\usepackage{hyperref}
\usepackage{xspace}
\usepackage[english]{babel}
\usepackage[autostyle]{csquotes}
\usepackage[a4paper, total={6in, 8in}]{geometry}
\usepackage{siunitx}
\usepackage{tikz}
\usetikzlibrary{shapes, arrows}
\usepackage[activate={true,nocompatibility},final,tracking=true,kerning=true,spacing=true,factor=1100,stretch=10,shrink=10]{microtype}
% activate={true,nocompatibility} - activate protrusion and expansion
% final - enable microtype; use "draft" to disable
% tracking=true, kerning=true, spacing=true - activate these techniques
% factor=1100 - add 10% to the protrusion amount (default is 1000)
% stretch=10, shrink=10 - reduce stretchability/shrinkability (default is 20/20)
\microtypecontext{spacing=nonfrench}

\setlength{\parindent}{0cm}

\def\realnumbers{\ensuremath{\mathbb{R}}\xspace}
\def\naturalnumbers{\ensuremath{\mathbb{N}}\xspace}
\def\complexnumbers{\ensuremath{\mathbb{C}}\xspace}
\newcommand{\SO}{\ensuremath{\mathit{SO}(3)}\xspace}
\newcommand{\SE}{\ensuremath{\mathit{SE}(3)}\xspace}
\newcommand{\myMatrix}[1]{\ensuremath{\begin{pmatrix}#1\end{pmatrix}}}
\newcommand{\myVector}[1]{\ensuremath{\begin{pmatrix}#1\end{pmatrix}}}
\newcommand{\quaternionen}{\ensuremath{\mathbb{H}}\xspace}

\begin{document}
	\pagenumbering{Roman}
	{\let\newpage\relax\maketitle}
	\tableofcontents
	\newpage
	\pagenumbering{arabic}
	\setcounter{page}{1}

\section{Canonical Ordering}
Every planar graph has a planar straigh-line drawing.
To construct such a drawing, we can use the \textit{Shift
algorithm} (aka \textit{De Fraysseix Pach Pollack algorithm}) or use a
\textit{Schnyder Realizer} to obtain a \textit{Barycentric Representation}.
No matter which approach we choose, we first need a \textbf{Canonical Ordering}
of the graph.

Assume that we have a \textbf{triangulated planar embedded graph} (embedding
of a maximal planar graph) $G = (V,E)$ with at least 3 vertices ($n \geq 3$).
On that, a \textbf{canonical ordering} is guaranteed to exist.

A canonical ordering is an ordering of vertices $\pi = (v_1, v_2, \cdots, v_n)$
for which the following conditions hold:

\begin{enumerate}
    \item We can remove an arbitrary amount of vertices in reverse order (=
        take off the last x vertices) and still end up with a 2-connected
        triangulated graph, as long as we still have at least 3 vertices left.
        More formally:

        For all $k \in \mathbb{N}$ with $3 \leq k \leq n$ the set $\{v_1,
        \cdots, v_k\}$ induces the 2-connected triangulated graph $G_k$.

    \item The first two vertices $v_1, v_2$ belong to the outer face of $G_k$.

    \item When adding vertices in ascending order, i.e. when iterating from
        $G_3$ to $G_n$, the new vertex $v_{k+1}$ always lies on the outer face
        of $G_k$ and all its neighbours lie on the outer boundary of $G_k$.

        This means that you can add the vertices in ascending order without
        creating crossings.
\end{enumerate}

\subsection{Computing Canonical Ordering}
In order to compute the canonical ordering, we use the following steps:

\begin{itemize}
    \item $out(v)$ means that \enquote{v lies on the outer boundary}.
        Intialized to false.
    \item $mark(v)$ means that \enquote{v has been checked}.
        Intialized to false.
    \item $chords(v)$ counts the \enquote{chords that v is attached to}.
        Intialized to 0.
\end{itemize}


\begin{enumerate}
    \item The first two vertices $v_1, v_2$ and the last vertex $v_n$ is set
        to be on the outer boundary, so $out(v_1), out(v_2), out(v_n) = true$.
    \item We iterate from n down to 3.
        Start with $k = n$.
    \item Choose an umarked vertex $v$ on the boundary that is not connected to
        a chord ($mark(v) = false, out(v) = true, chords(v) = 0$).
    \item Mark $v$ ($mark(v) = true$) and set v as the $k$-th element of the
        canonical order ($v_k = v$).
    \item Remove $v$ and list the vertices of the new boundary (= of the new
        graph $G_{k-1}$) as $w_1, \cdots, w_t$ with $w_1 = v_1, w_t = v_2$.

        A subsequence of this new ordering are the unmarked neighbours of
        $v_k$.
        We will name them $w_p, \cdots, w_q$.
    \item The \textbf{inner} neighbours (excluding $w_p, w_q$) have just been
        exposed to the outer face and are now on the boundary.
        Set $out(w_i) = true$ for $p < i < q$.
    \item Update the amount of chords for all $w_i$ and its neighbours.
    \item $k++$. Go back to step 3 unless $k<3$.
\end{enumerate}

\textbf{Runtime}: $\mathcal{O}(n)$

\section{Shift Algorithm}

We start with the following positions:

\begin{itemize}
    \item $v_1$ on $(0,0)$
    \item $v_2$ on $(2,0)$
    \item $v_3$ on $(1,1)$
\end{itemize}

Now we add the vertices in ascending canonical order as follows:

\begin{enumerate}
    \item $k = 4$
    \item Identify the already added neighbours of $v_k$.
    \item The first neighbour and all vertices left of it are not moved.
    \item The last neighbour and all vertices right of it are moved by $+2$ in
        x-direction.
    \item The inner neighbours are moved by $+1$ in x-direction.
    \item Draw a north-east (north-west) guideline from the first (last) neighbour
        with $slope = 1$ ($slope = -1$).
    \item Add $v_k$ on the intersection of these lines.
        Remove the guidelines afterwards.
    \item $k++$. Go back to step 2 unless $k>n$.
\end{enumerate}

When finished, the positions of $v_1, v_2, v_n$ are always the same:
\begin{itemize}
    \item $v_1$ on $(0,0)$
    \item $v_2$ on $(2n-4,0)$
    \item $v_n$ on $(n-2,n-2)$
\end{itemize}

\textbf{Invariants}:
For $G_{k-1}$:
\begin{itemize}
    \item $v_1$ is on $(0,0)$ and $v_2$ is on $(2k-6,0)$
    \item The boundary (without edge $(v_1, v_2)$) is x-monotone (e.g. always
        goes to the right).
    \item Each edge on the boundary (without edge $(v_1, v_2)$) has $slope =
        \pm1$.
\end{itemize}

\textbf{Area}: $(2n-4) * (n-2)$

\section{Barycentric Representation}
A \textbf{Barycentric Representation} of a graph $G = (V,E)$ is an assignment
of barycentric coordinates to each of G's vertices.
It is an \textit{injective} function $v \mapsto (v_a, v_b, v_c)$ such that

\begin{itemize}
    \item The sum of the coordinates is always 1:\\
        $v_a + v_b + v_c = 1$
    \item If we have an edge $(x, y)$ and a third vertex $z$, then this vertex
        is always closer to $A$, $B$, or $C$ than both $x$ or $y$, formally:\\
        $\exists k \in \{a, b, c\}$ with $x_k < z_k$ and $y_k < z_k$.
\end{itemize}

Drawing $G$ in the triangle $\Delta A B C$ with its barycentric representation
will \textbf{always} result in a planar drawing.

\subsection{Schnyder Labelling}
Before we get to the \textit{Schnyder Realizer} with which we can create a
barycentric representation, we have to create a \textit{Schnyder Labelling}.

A Schnyder Labelling of a planar triangulated graph $G$ is a labelling of all
internal angles (angles not on the outer face) with label 1, 2, and 3 such
that:
\begin{itemize}
    \item Each internal face contains all three labels in counterclockwise
        order.
        (Remember that all internal faces in a triangulated graph are
        triangles.)
    \item Each vertex is surrounded by angles of all three kinds, appearing in
        counterclockwise order.
        Note that a vertex might have more than three labels attached to it.
        If that is the case, duplicate labels appear next to each other.
        So at each vertex, you have a \textit{nonempty} interval of 1's, then
        2's, then 3's.
\end{itemize}

\textbf{Every triangulated plane graph} has a Schnyder labelling!

More important properties:
\begin{itemize}
    \item The labels at each of the triangle points $A, B, C$ are the same,
        e.g. all labels at $A$ are 1.
    \item Each edge induces 4 labels (two angles per vertex).
        At one end of each edge, the labels are the same.
        You can characterize an edge by the name of the duplicate label and the
        end on which those labels appear.
\end{itemize}

If we say that each edge points in the direction of the duplicate label, we
can define a\ldots

\subsection{Schnyder Realizer}

A \textbf{Schnyder Realizer} or \textbf{Schnyder Forest} partitions the edges
of a triangulated planar graph $G$ into three sets $T_1, T_2, T_3$.
Edges pointing towards a duplicate \enquote{1} label belong to $T_1$ and so on.

This partitioning also has to satisfy the following constraints for each inner
vertex $v$:
\begin{itemize}
    \item $v$ has exactly one outgoing edge in each of $T_1, T_2, T_3$.
    \item The edges appear in six intervals, similar to the intervals in the
        Schnyder labelling: Outgoing in $T_1, T_2, T_3$ and incoming in $T_1,
        T_2, T_3$.

        The outgoing interval is always opposite of the incoming interval of
        the same label.
        The incoming intervals appear in counterclockwise order.

        So one could say the order of intervals is:\\
        $T_1$ incoming, $T_3$ outgoing, $T_2$ incoming, $T_1$ outgoing, $T_3$
        incoming, $T_2$ outgoing.
\end{itemize}

\textbf{Every triangulated plane graph} has a Schnyder realizer!

\subsection{From Schnyder Realizer to Barycentric Representation}

Let's assign colours to the edge sets of the Schnyder Realizer:\\
\textcolor{red}{$T_1$} is red, \textcolor{blue}{$T_2$} is blue,
\textcolor{green}{$T_3$} is green.

From this we can deduce a few interesting properties:
\begin{itemize}
    \item $A, B, C$ are the only sinks for the red / blue / green paths.
    \item From each vertex, there is a red / blue / green path to $A / B / C$.
        Let's call it $P_a(v) / P_b(v) / P_c(v)$.
    \item No monochromatic cycle exists.
    \item Each monochromatic subgraph is a tree!
\end{itemize}

The Paths $P_a(v) / P_b(v) / P_c(v)$ cross only at $v$ and thus divide the
graph into three face sets:

The set of faces $R_a(v)$ lies on the opposite side of $A$, bound by $P_b(v),
P_c(v)$ and $(B, C)$.
$R_b(v)$ and $R_c(v)$ are defined similarly.

If another vertex $u$ lies in one of the sets of faces induced by $v$, called
$R_i(v)$, $u$'s set of faces on that side ($R_i(u)$) is a strict subset of
$R_i(v)$.

The \textbf{barycentric coordinates} of a vertex $v$ are
$$ f: v \mapsto (v_a, v_b, v_c) = \frac{1}{2n-5} (|R_a(v)|,|R_b(v)|,|R_c(v)|) $$

The function $f$ is the \textbf{barycentric representation} of $G$.

\textit{Note}: $2n-5$ is the total number of faces in $G$.
Normalizing by that guarantees that the sum of the coordinates is always 1.

\textbf{Area}: $(2n-5)^2$

\subsection{Weak Barycentric Coordinates}

We can make the graph even smaller by using \textbf{weak barycentric
coordinates}:
$$ f: v \mapsto (v_a, v_b, v_c) = \frac{1}{n-1} (|n_1(v)|,|n_2(v)|,|n_3(v)|) $$
$$ n_i(v) = |\text{vertices in }R_i(v)| - |P_{i-1}(v)| $$
$$ A = (n-1, 0), B = (0, n-1), C = (0,0) $$

The second condition differs a bit from standard barycentric coordinates:

\begin{itemize}
    \item For each edge $(u, v)$ and third vertex ($z$):
        $$ \exists k \in \{a, b, c\} \text{ so that }
        (u_k, u_{k+1}) <_{lex} (z_k, z_{k+1}) \text{ and }
        (v_k, v_{k+1}) <_{lex} (z_k, z_{k+1}) $$
\end{itemize}

\textbf{Area}: $(n-2)^2$

\section{Orthogonal Drawing}

\subsection{\textit{st}-ordering}
An \textbf{\textit{st}-ordering} is an ordering of vertices $\{v_1, v_2, \dots,
v_n\}$ such that all inner vertices (all but $v_1, v_n$) are adjacent to a
preceding and a following vertex.
More formally: For each $j, 2 \leq j \leq n-1$ the vertex $v_j$ hat at least
one neighbour $v_i$ with $i < j$ and at least one neighbour $v_k$ with $j < k$.

The first node $v_1$, also called $s$, is the only source in the graph.
The last node $v_n$, also called $t$, is the only sink.

\textbf{Every biconnected graph} has an \textit{st}-ordering.

\subsection{Biedl \& Kant Algorithm}
The Biedl \& Kant algorithm assigns a drawing pattern to each vertex, based on
its indegree (see slides). The first vertex is drawn differently.

Based on the pattern, vertices may create a number of columns:
\begin{itemize}
    \item First vertex: At most 3
    \item Vertex with indegree 1: At most 2
    \item Vertex with indegree 2: At most 1
    \item Vertex with indegree 3 or 4: No new columns
\end{itemize}

\textbf{Area}: Width $m - n + 1$, height at most $n+1$

\textbf{Amount of bends}: At most $2m - 2n + 4$

\textbf{Runtime}: $\mathcal{O}(n)$ if an \textit{st}-ordering is provided

\textbf{Bends per edge}: At most 2 (except for maybe one edge with 3 bends)

\textbf{If $G$ is plane}, the orthogonal drawing is planar.

\section{Constructing an \textit{st}-ordering}
If we have directed acyclic graph $G$ with one source and one sink, then we
call it an \textbf{\textit{st}-digraph}.

On this \textit{st}-digraph, we construct a \textbf{topological ordering}
$v_1, v_2, \dots, v_n$ such that for every edge, the parent vertex comes before
the child vertex.
Formally: $\forall (v_i, v_j) \in E: i < j$

\subsection{Orienting edges}
If we start with an undirected biconnected graph $G$, we first have to orient
the edges of $G$.
For that, we use \textbf{ear decomposition}.

An ear decomposition is a partition of the edges of $E$ into an ordered
collection of disjoint paths $P_0, \dots, P_r$ with the following properties:

\begin{itemize}
    \item $P_0$ is just a single edge.
    \item $P_0 \cup P_1$ is a simple cycle.
    \item All paths $P_i$ meet the previous paths in their end-vertices, i.e.
        the endpoints also belong to $P_1 \cup \dots \cup P_{i-1}$
    \item No internal vertices of a path belong to previous paths.
\end{itemize}

\textbf{Every biconnected graph} $G$ with an edge $(s, t)$ has an open ear
decomposition with $P_0 = (s, t)$.

With this ear decomposition we can now create an \textit{st}-orientation:
\begin{enumerate}
    \item Assign a direction to $P_0$.
        $P_0$ now belongs to the set of directed ears.
    \item If two vertices are in the set of directed ears that are also
        connected by an undirected ear, i.e. are the end-points of another ear,
        apply the same direction to that other ear (in a way that does not
        produce cycles).
    \item Repeat step 2 until all ears are oriented.
\end{enumerate}

The resulting graph $G'$ is the \textbf{\textit{st}-orientation} of $G$.

\subsection{Topological ordering}
With this \textit{st}-orientation, we can create a topological ordering as
follows:
\begin{enumerate}
    \item For $P_0$ and $P_1 = \{u_1, \dots, u_p\}$, the end-vertices are the
        source $s$ and sink $t$.
        The sequence $L = \{u_1, \dots, u_p\}$ is an \textit{st}-ordering of
        $G_1$.
    \item From here on, take the next ear $P_{i+1} = \{v_1, \dots, v_p\}$ and
        insert it into $L$ between $v_1$ and $v_p$.
        Note that $v_1$ and $v_p$ are the end-vertices of $P_{i+1}$ but are
        also contained in the previous ears so they're already in $L$.
    \item Repeat step 2 until all vertices are added.
\end{enumerate}

\textbf{Runtime}: $\mathcal{O}(E)$

\end{document}
